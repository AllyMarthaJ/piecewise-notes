\section{Notation}
\subsection{Single variable piecewise functions}
A piecewise \textit{function} is a function defined over several pieces. 

Let us consider some function $f:\mathbb{R}\to\mathbb{R}$ and some intervals ${D_1,D_2,\dots,D_n\subseteq\mathbb{R}}$ such that ${D_1\cap D_2\cap \dots\cap D_n=\varnothing}$. Suppose we have some functions $f_1: D_1\to\mathbb{R}$ and so on. We describe this function using the following notation:

$$
    f(x) = \begin{piecewise} 
        f_1(x) & x\in D_1 \\
        f_2(x) & x\in D_2 \\
        \vdots & \vdots \\
        f_n(x) & x\in D_n
    \end{piecewise}
$$

This same function could instead be written as the following:

\begin{align*}
    f:D_1\to\mathbb{R}&, f(x)=f_1(x), \\
    f:D_2\to\mathbb{R}&, f(x)=f_2(x), \\
    \vdots \\
    f:D_n\to\mathbb{R}&, f(x)=f_n(x)
\end{align*}

This essentially describes the same function over different domains. Alternatively, one might think about it as different functions under the same label, $f$.

Reading this function more verbosely, we say that if $x$ is in the domain $D_1$ and so on, then we might `choose' to have $f(x)=f_1(x)$ as per our definition.

\begin{example}[Absolute value function]
    Let us define the absolute value function, $\abs{x}$ as the following:

    $$
        \abs{x} = \begin{piecewise}
            x & x\geq 0 \\
            -x & x\leq 0
        \end{piecewise}
    $$

    To evaluate, for example, $\abs{-3}$, we use the definition:

    $$
        \abs{-3} = \begin{piecewise}
            -3 & -3\geq 0 \\
            3 & -3\leq 0
        \end{piecewise}
    $$

    Since the condition in the second piece (or case) evaluates to true, we use the value in that piece. That is, since $-3\leq 0$ is true, $\abs{-3}=3$.
\end{example}

\subsection{Piecewise functions}
Piecewise functions needn't be restricted to a single variable. More generally, piecewise functions needn't be restricted to any number of conditions or pieces.

\newpage{}