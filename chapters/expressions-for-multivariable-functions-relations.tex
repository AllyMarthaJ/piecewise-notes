% want to definitely cover the shapes stuff
% want to start off on a low level - how we might approach sticking, top to bottom, two functions together
% original post
% we should cover, squares, diamonds, triangles, batman equation, as examples or highlights.
\section{Expressions for Relations; Multivariable Functions}
We've previously covered the idea of `gluing' several functions together along a real number line, or through some condition, in one dimension. We can bring this idea to higher dimensions; functions in several variables with gluing along arbitrary planes, lines or other functions. There are infinitely many ways to do this, and with the tools set out in these notes, the problems become less of `how do I express this function in terms of elementary functions?' and more of a modelling problem.

In any case, in this section we cover a few select cases on gluing functions together. In particular, we investigate and use the behaviour of intersecting these functions with a plane to create expressions for arbitrary shapes, and investigate shapes.

\subsection{Introduction to curves generated by gluing functions}
Suppose we have two functions, $f(x,y)$ and $g(x,y)$, and we want to glue them along the curve $f(x,y)=g(x,y)$. We could do this in a few ways, two of which are:
$$
    s(x,y)=\begin{piecewise}
        f(x,y) & f(x,y)\geq g(x,y) \\
        g(x,y) & f(x,y)\leq g(x,y)
    \end{piecewise}=\max\br{f(x,y),g(x,y)}
$$

or:

$$
    s(x,y)=\begin{piecewise}
        g(x,y) & f(x,y)\geq g(x,y) \\
        f(x,y) & f(x,y)\leq g(x,y)
    \end{piecewise}=\min\br{f(x,y),g(x,y)}
$$

This actually encodes some fairly complex behaviour in our functions (for example solving for $f(x,y)=g(x,y)$ is not easy); perhaps this is not what we want. Perhaps this is a backwards way of defining what we want, but for the sake of argument, let us now suppose that we have $f(x,y)=y-F(x)$ and $g(x,y)=G(x)-y$, and suppose $s(x,y)=0$. That is, from our first formulation:

\begin{align*}
    s(x,y) &= \begin{piecewise}
        y-F(x) & y-F(x)\geq G(x)-y \\
        G(x)-y & y-F(x)\leq G(x)-y
    \end{piecewise} = 0
\end{align*}

Suddenly, expressing this formulation seems a lot harder, even if we know we can express it by $s(x,y)=\max\br{y-f(x),G(x)-y}=0$. But what does it actually mean?

Let us begin from scratch. We've derived a way to glue two functions in two arguments in a higher dimension; this is necessarily a continuous surface in $\mathbb{R}^3$. We then took an intersection for which $s(x,y)=0$, which doesn't necessarily produce a continuous curve in $\mathbb{R}^2$. It stands to reason, therefore, that we might use this to connect two graphs in $\mathbb{R}^2$ which can be expressed with $y=\dots$.

Let us take the graphs of two functions $f(x)$ and $g(x)$; $y=f(x)$ and $y=g(x)$, and their subsequent `arithmetic mean'; $y=\frac{f(x)+g(x)}{2}$. We wish to plot $f(x)$ if $y$ is `above' this curve and $g(x)$ otherwise. That is:
$$
    y = \begin{piecewise}
        f(x) & y\geq \frac{f(x)+g(x)}{2} \\
        g(x) & y\leq \frac{f(x)+g(x)}{2}
    \end{piecewise}
$$

This piecewise function is not well-defined, since it involves recursion on $y$ (i.e. substitutions on $y$ won't work). Instead, we attempt to manipulate it such that we have a single expression. This is done by first substracting the piecewise object from both sides and moving $y$ in, then multiplying the latter piece by $-1$ since both pieces vanish:

$$
    \begin{piecewise}
        y-f(x) & y\geq \frac{f(x)+g(x)}{2} \\
        g(x)-y & y\leq \frac{f(x)+g(x)}{2}
    \end{piecewise} = 0
$$

Using our gluing formula, we have:
$$
    \paren*{\max\br*{y,\frac{f(x)+g(x)}{2}}-f(x)}+\paren*{g(x)-\min\br*{y,\frac{f(x)+g(x)}{2}}}-\paren*{\frac{g(x)-f(x)}{2}}=0
$$

Simplifying gives us:
$$
    \max\br*{y,\frac{f(x)+g(x)}{2}}-\min\br*{y,\frac{f(x)+g(x)}{2}}=\frac{f(x)-g(x)}{2}
$$

Which, using the identity $\max\br{a,b}-\min\br{a,b}=\abs{a-b}$, gives:
$$
\abs{y-\frac{f(x)+g(x)}{2}}=\frac{f(x)-g(x)}{2}
$$

What should be noted is that this gives the beginnings of creating closed shopes. For example, take $f(x)=1-x^2$ and $g(x)=x^2-1$, which gives $\abs{y}=1-x^2$. We only plot both $y=1-x^2$ and $y=x^2-1$ when $1-x^2\geq 0$ and $x^2-1\leq 0$; creating a closed shape curve (namely $\br{(x,y)\in\mathbb{R}^2\mid \abs{y}=1-x^2}$).

\subsection{Curves generated by intersections of boundary curves}
\subsubsection{Logical `or' and `and' properties generating curves}
Recall that we can `combine' equality conditions using \ref{theorem:logical_and_equality} and \ref{theorem:logical_or_equality}. Furthermore, recall that $x\in\brak{a,b}\iff \ell_{a}^{b}(x)=x$. Using these two tools, we can create curves using a mix of interpolation and clever modelling.

Suppose that we have a function $F:\mathbb{R}^2\to\mathbb{R}$ such that we have a curve given by $F(x,y)=0$. Let us also define $f_1,f_2,\dots,f_n:\mathbb{R}^2\to\mathbb{R}$ such that $f_1,f_2,\dots,f_n$ define the boundary of the curve we want to create. Namely let us also define $g_1,g_2,\dots,g_n$ such that $g_k=0$ when $f_k$ `contributes' to the boundary.

We construct $F$ as follows:
$$
    F(x,y) = \begin{piecewise}
        0 & f_1(x,y)=0 & g_1(x,y)=0 \\
        0 & f_2(x,y)=0 & g_2(x,y)=0 \\
        \vdots & \vdots & \vdots \\
        0 & f_n(x,y)=0 & g_n(x,y)=0 \\
        \star & \star & \star
    \end{piecewise} = r(x,y)\prod_{k=1}^{n}\paren*{p_{k,1}(f_k(x,y))+p_{k,2}(g_k(x,y))}
$$

For some arbitrary, non-zero, function $r(x,y)$, as well as $p$ functions $p_{k,1}, p_{k,2}$.

Since $F(x,y)=0$ forms our desired curve, we have:
$$
    r(x,y)\prod_{k=1}^{n}\paren*{p_{k,1}(f_k(x,y))+p_{k,2}(g_k(x,y))}=0
$$

Creating the expressions for $g_{k}$ is fairly straightforward for inequalities. As an alternative for $\ell_{a}^{b}(x)=x$, we can also use $\max\br*{\abs{x-\frac{a+b}{2}}+\frac{a-b}{2},0}=0$; this is a direct result of $-m\leq x\leq m\iff \abs{x}\leq m$ and $x\leq 0\iff \max\br{x,0}=0$.

\begin{example}
    We want to find a curve such that for all $(x,y)\in\mathbb{R}^2$, we have $y=x$ for $x\geq 0$ and $y=-x$ for $x\leq 0$.

    Creating a function to generate the curve, we have:
    $$
        F(x,y) = \begin{piecewise}
            0 & y-x & x\geq 0 \\
            0 & y+x & x\leq 0 \\
            \star & \star & \star
        \end{piecewise}
    $$

    Recall $x\geq 0\iff\abs{x}=x$ and $x\leq 0\iff\abs{x}=-x$. Using $\abs{x}$ as our $p$ functions therefore, we get:
    $$
        F(x,y) = \begin{piecewise}
            0 & \abs{y-x}+\abs{\abs{x}-x}=0 \\
            0 & \abs{y+x}+\abs{\abs{x}+x}=0 \\
            \star & \star
        \end{piecewise} = (\abs{y-x}+\abs{\abs{x}-x})(\abs{y+x}+\abs{\abs{x}+x})
    $$

    So our final curve is:
    $$
        \br*{(x,y)\in\mathbb{R}^2\mid (\abs{y-x}+\abs{\abs{x}-x})(\abs{y+x}+\abs{\abs{x}+x})=0}
    $$
\end{example}

This method has a pitfall in that it is unable to be graphed on most graphing calculators, as they make use of certain optimisations which interfere with the graphing of them. A good, simple example of this is the curve $\br*{(x,y)\in\mathbb{R}^2\mid \abs{y^2-x^2}=0}$. This curve is equivalent to $\br*{(x,y)\in\mathbb{R}^2\mid (y=x)\lor (y=-x)}$, but fails to be graphed sufficiently in tools like Mathematica, Desmos, and GeoGebra.

These graphs can be approximated, however; regard that our function describing the curve, $s(x,y)$, generates our curve exactly for $s(x,y)=0$; it stands to reason that in a neighbourhood of $0$ we should have something `similar' to our curves (since $s(x,y)$ is continuous). That is, for some small $\varepsilon>0$ we have that $s(x,y)=\varepsilon$ approximates our curve.

\subsubsection{Max of boundary functions generating curves}
In the previous method of creating curves, we explicitly specified the boundary intersection points in our piecewise functions, which allowed us to create curves that are, for all intents and purposes, human readable, at the cost of being lengthy. Using the $\max$ function for boundary functions, we describe our boundary conditions in terms of other boundaries. The idea behind this is so that we can succinctly describe the boundary in terms of a single $\max$ expression.

\begin{example}
    For the boundary $y=0$ and conditions $0\leq x\leq 1$, we can define the boundaries $x=0$ and $x=1$. That is,
    $$
        F(x,y)=\begin{piecewise}
            0 & y=0 & x\geq 0 & x\leq 1
        \end{piecewise}
    $$

    In order for our definition to match the $\max$ function, we should have $x\geq 0\iff -x\leq 0$ and $x\leq 1\iff x-1\leq 0$. Then using the substitution $y=0$, we have:
    $$
        F(x,y)=\begin{piecewise}
            y & y=0 & -x\leq y & x-1\leq y
        \end{piecewise}
    $$

    Similarly, we could also write:
    $$
        F(x,y)=\begin{piecewise}
            -y & -y=0 & -x\leq -y & x-1\leq -y
        \end{piecewise}
    $$

    These generate two similar curves for $F(x,y)=0$ but are not identical (it depends on other boundary conditions). In any case, we can eliminate the $y=0$ or $-y=0$ condition if and only if the entire function is $0$. This gives us either:
    $$
        F(x,y)=\max\br*{-x,x-1,y}=0
    $$
    or
    $$
        F(x,y)=\max\br*{-x,x-1,-y}=0
    $$
\end{example}

Using the above example as motivation for a general case, let us define the boundary functions $f_1,f_2,\dots,f_n:\mathbb{R}^2\to\mathbb{R}$ so that we construct the function:
$$
    F(x,y) = \begin{piecewise}
        0 & f_1(x,y) = 0 & f_2(x,y)\leq 0 & f_2(x,y)\leq 0 & \dots & f_n(x,y) \leq 0 \\
        0 & f_2(x,y) = 0 & f_1(x,y)\leq 0 & f_3(x,y)\leq 0 & \dots & f_n(x,y) \leq 0 \\
        \vdots & \vdots & \vdots & \vdots & \ddots & \vdots \\
        0 & f_n(x,y) = 0 & f_1(x,y)\leq 0 & f_2(x,y)\leq 0 & \dots & f_{n-1}(x,y) \leq 0 \\
    \end{piecewise}
$$

For all $k\in\br{1,2,\dots,n}$, $f_k(x,y)=0$, we can use the substitution $f_k(x,y)=0$ in the $k$ piece such that we have:
$$
    F(x,y) = \begin{piecewise}
        f_1(x,y) & f_1(x,y) = 0 & f_2(x,y)\leq f_1(x,y) & f_2(x,y)\leq f_1(x,y) & \dots & f_n(x,y) \leq f_1(x,y) \\
        f_2(x,y) & f_2(x,y) = 0 & f_1(x,y)\leq f_2(x,y) & f_3(x,y)\leq f_2(x,y) & \dots & f_n(x,y) \leq f_2(x,y) \\
        \vdots & \vdots & \vdots & \vdots & \ddots & \vdots \\
        f_n(x,y) & f_n(x,y) = 0 & f_1(x,y)\leq f_n(x,y) & f_2(x,y)\leq f_n(x,y) & \dots & f_{n-1}(x,y) \leq f_n(x,y) \\
    \end{piecewise}
$$

Again, we can now eliminiate the $f_k(x,y)=0$ case if and only if the entire function is zero. Using the definition of the $\max$ function then, we get:
$$
    F(x,y)=\max\br*{f_1(x,y),f_2(x,y),\dots,f_n(x,y)}=0
$$

\begin{example}
    We wish to find the boundary functions of a triangle with vertices $(-1,-1)$, $(1,-1)$, $(0,1)$.

    From these vertices, we get three linear lines for $(x,y)\in\mathbb{R}^2$: $y=-1$, $1-2x$ and $y=2x+1$.

    Regard now that we wish to have the following:
    \begin{itemize}
        \item $y\geq -1\implies -y-1\leq 0$,
        \item $y\leq 1-2x\implies y+2x-1\leq 0$,
        \item $y\leq 2x+1\implies y-2x-1\leq 0$
    \end{itemize}

    Therefore, it stands to reason that our boundary functions are:
    \begin{itemize}
        \item $f_1(x,y)=-y-1$,
        \item $f_2(x,y)=y+2x-1$,
        \item $f_3(x,y)=y-2x-1$
    \end{itemize}
\end{example}

% proof that the set of convex polygons can be given by these?
% convex polygons <-> max

\subsection{Regions generated by intersections of boundary curves}
\subsubsection{Logical `or' and `and' properties generating regions}
\subsubsection{Max of boundary functions generating regions}


\newpage