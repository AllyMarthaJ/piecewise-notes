\section{Further Piecewise Algebra}
Until now, we've been focusing on ways we can use existing functions to motivate forms for our own constructions. It turns out there are far simpler methods of deriving what we have before using an interpolation-style approach with piecewise notation (which do in fact yield two important interpolation polynomials; Newton polynomials and Lagrange polynomials).

\subsection{Anonymous piecewise objects}
The name anonymous piecewise object is more or less misleading, but we use these objects as a basis for extending the definition of a piecewise object where not defined. Namely, given a piecewise object, we can formulate an expression which defines the behaviour of the piecewise object overall.

When we describe an anonymous piecewise object, we're actually describing a (very infinite) set of possible objects which we can use to represent an existing piecewise object. We simply denote this by $\star$ inside a piecewise object for either conditions, piece values or both (though when in a condition, it's effectively equivalent to the condition `otherwise').

\begin{example}
    \label{example:anonymous_piecewise_1}
    We wish to denote a function $f:\mathbb{R}\to\mathbb{R}$ such that $f(0)=0$ and $f(1)=1$ using an anonymous piecewise function. We do so by writing the following:

    $$
        f(x) = \begin{piecewise}
            0 & x=0 \\
            1 & x=1 \\
            \star & \star
        \end{piecewise}
    $$

    This represents a set of functions $f$ such that the above is satisfied. Also, by observation, a function $f(x)=x$ would satisfy this.

    We can define, explicitly, $f(x)\in S$ where $S=\br*{f\in \mathcal{F}\mid f(0)=0\land f(1)=1}$ where $\mathcal{F}$ is the set of all functions from $\mathbb{R}$ to $\mathbb{R}$.

\end{example}

Any time we start manipulating such objects as will be discussed in this section, we reduce the number of functions that satisfy the piecewise definition provided. Generally, anonymous piecewise objects will be denoted (if the respective piecewise object is $\phi$) $\phi^\star$. This will be explored further. However, for the purposes of most of the working here, we \textit{minimally} interpolate.

\subsection{Polynomial extraction}
We should distinguish initially between the sort of `extractions' we can perform on piecewise objects. The first of which is polynomial extraction, as we rely on the roots of a polynomial. In our case, these roots are formed by the pieces of our piecewise objects.

The premise of this extraction is as follows: We can reduce the number of pieces of a piecewise object by vanishing any number of those pieces, and `extracting'. This extraction is done by means of multiplication and addition on the respective piece conditions.
\begin{example}
    Let us use the logical NOT function; that is, $f:\br{0,1}\to\br{0,1}$:
    $$
        f(x) = \begin{piecewise}
            1 & x = 0 \\
            0 & x = 1
        \end{piecewise}
    $$

    Recall that previously we would have manipulated the piece values to match the conditions so as to simplify this piecewise function. Instead, we're going to `extract' the root corresponding to $x=1$; $x-1$ (since that piece is $0$):
    $$
        f(x) = (x-1)\begin{piecewise}
            \frac{1}{x-1} & x=0 \\
            0 & x=1
        \end{piecewise}
    $$

    Since the entire expression is $0$ when $x=1$ outside of the piecewise object, we can remove the piece when $x=1$ inside the piecewise object (adding in $\star$ to denote all other pieces), thereby leaving us with:
    $$
        f(x) = (x-1)\begin{piecewise}
            \frac{1}{x-1} & x=0 \\
            \star & \star
        \end{piecewise}
    $$

    Evaluating the piece value for when $x=0$ (using a substitution for $x$), we now have:
    $$
        f(x) = (x-1)\begin{piecewise}
            -1 & x=0 \\
            \star & \star
        \end{piecewise}
    $$

    Since there's only one piece we can use left, we reduce:
    $$
        f(x) = (x-1)(-1)
    $$

    And so $f(x)=1-x$. Or rather, $1-x\in S$ where $S$ is the set of all functions from $\br{0,1}\to\br{0,1}$ (though for simplicity we shall just say the former going forward).

    As a disclaimer, this is only one possible function that can satisfy the above conditions. The reason we don't have all possible relevant functions as the result is because when we reduced our pieces (not when we extracted the root, because they are equivalent forms), we eliminated the $x=1$ piece explicitly. Likewise, when we reduced the final piecewise object down to a constant, we eliminated any remaining cases.
\end{example}

\begin{example}
    \label{example:max_further}
    We revisit the absolute value formula for $\max$ once again. Recall the piecewise definition of this function per \ref{section:max}.

    We wish to derive the absolute value formula for $\max$:
    $$
        \max\br{a,b}=\begin{piecewise}
            a & a\geq b \\
            b & a\leq b
        \end{piecewise}
    $$

    Recall that $a\geq b\iff \abs{a-b}=a-b$ and $a\leq b\iff \abs{a-b}=b-a$. We perform this substitution on these conditions, leaving us with:
    $$
        \max\br{a,b}=\begin{piecewise}
            a & \abs{a-b}=a-b \\
            b & \abs{a-b}=b-a
        \end{piecewise}
    $$

    Subtracting $b$ out and `extracting' the second piece, we get:
    $$
        \max\br{a,b}=b+\paren*{\abs{a-b}+a-b}\begin{piecewise}
            \frac{a-b}{\abs{a-b}+a-b} & \abs{a-b}=a-b \\
            \star & \star
        \end{piecewise}
    $$

    Evaluating the remaining piece value, we have:
    $$
        \max\br{a,b}=b+\paren*{\abs{a-b}+a-b}\begin{piecewise}
            \frac{a-b}{2(a-b)} & \abs{a-b}=a-b \\
            \star & \star
        \end{piecewise}
    $$

    Simplifying:
    \begin{align*}
        \max\br{a,b} &= b+\frac{1}{2}\paren*{\abs{a-b}+a-b} \\
        &= \frac{1}{2}\paren*{a+b+\abs{a-b}}
    \end{align*}
\end{example}

\begin{example}
    \label{example:heaviside_further}
    Let us revisit the Heaviside step function:
    $$
        \step{x} = \begin{piecewise}
            1 & x > 0 \\
            0 & x < 0
        \end{piecewise}
    $$

    To rewrite this function in terms of $\sgn{x}$, recall that $x>0\iff \sgn{x}=1$ and likewise $x<0\iff \sgn{x}=-1$:
    $$
        \step{x} = \begin{piecewise}
            1 & \sgn{x}=1 \\
            0 & \sgn{x}=-1
        \end{piecewise}
    $$

    Extracting the second piece, we have:
    $$
        \step{x} = (\sgn{x}+1)\begin{piecewise}
            \frac{1}{\sgn{x}+1} & \sgn{x} = 1\\
            \star & \star
        \end{piecewise}
    $$

    Evaluating the remaining piece and simplifying gives:
    $$
        \step{x} = \frac{\sgn{x}+1}{2}
    $$

    An interesting byproduct of this is that should $\step{0}$ be defined, this interpolation would give $\step{0}=\frac{1}{2}$.

    Furthermore if $\abs{x}$ were used instead of $\sgn{x}$ we would have that:
    $$
        \step{x} = \frac{\abs{x}+x}{2x}
    $$
\end{example}

While we can continue to extract pieces progressively and do the appropriate manipulation to simplify, there are 2 slightly different processes we can perform which yield Newton and Lagrange polynomials.

\subsubsection{Newton polynomials}
Let $f:\mathbb{R}\to\mathbb{R}$ be a function such that $f(x_m)=y_m$ for $m\in\br{0,1,2,\dots,n}$. Then $f$ can be given by a Newton polynomial:
$$
    f(x) = \sum_{m=0}^{n}{[y_0,\dots,y_m]\prod_{l=0}^{m-1}{(x-x_l)}}
$$

Where $[y_0,\dots,y_m]$ denotes divided differences. That is, $[y_0]=y_0$, $[y_0,y_1]=\frac{y_1-y_0}{x_1-x_0}$, $[y_0,y_1,y_2]=\frac{[y_1,y_2]-[y_0,y_1]}{x_2-x_0}$, and so on.

\begin{theorem}
    A Newton polynomial is formed by nesting pieces inside a piecewise object and recursively interpolating. Setting up the interpolation problem, we have:
    $$
        f(x) = \begin{piecewise}
            y_0 & x=x_0 \\
            y_1 & x=x_1 \\
            \vdots & \vdots \\
            y_n & x=x_n \\
            \star & \star
        \end{piecewise}
    $$

    Let us denote $f(x)=f_{n}(x)$ and then nest pieces $x=x_0,x_2,\dots,x_{n-1}$:
    $$
        f(x) = \begin{piecewise}
            \begin{piecewise}
                y_0 & x=x_0 \\
                y_1 & x=x_1 \\
                \vdots & \vdots \\
                y_{n-1} & x=x_{n-1} \\
                \star & \star
            \end{piecewise} & (x-x_0)(x-x_1)\dots(x-x_{n-1})=0\\
            y_n & x=x_n \\
            \star & \star
        \end{piecewise}
    $$

    How we achieved the condition on the nested piecewise function will be discussed later. Let us denote the nested piecewise function $f_{n-1}(x)$ such that:
    $$
        f_{n}(x) = \begin{piecewise}
            f_{n-1}(x) & (x-x_0)(x-x_1)\dots(x-x_{n-1})=0 \\
            y_{n} & x=x_n \\
            \star & \star
        \end{piecewise}
    $$

    Now, we have a recurrence relation in $f_{n}(x)$, given that $f_{0}(x)=y_0$ (since we a constant function satisfies the definition in this case). Extracting the first piece, containing the recurrence, we have:
    $$
        f_{n}(x)=f_{n-1}(x)+\paren*{\prod_{m=0}^{n-1}{(x-x_m)}}\begin{piecewise}
            \frac{y_{n}-f_{n-1}(x)}{\prod_{m=0}^{n-1}{(x-x_m)}} & x=x_{n} \\
            \star & \star
        \end{piecewise}
    $$

    Evaluating the last piece value and simplifying, we get:
    \begin{align*}
        f_{n}(x) &=f_{n-1}(x)+\frac{y_{n}-f_{n-1}(x_{n})}{\prod_{m=0}^{n-1}{(x_{n}-x_m)}}\prod_{m=0}^{n-1}{(x-x_m)} \\
        f_{0}(x) &= y_0
    \end{align*}
\end{theorem}

\begin{example}
    We wish to find a Newton polynomial $f$ corresponding to $f(0)=0$, $f(1)=1$ and $f(2)=4$.

    We begin by writing this problem piecewise:
    $$
        f(x) = \begin{piecewise}
            0 & x=0 \\
            1 & x=1 \\
            4 & x=2 \\
            \star & \star
        \end{piecewise}
    $$

    We now nest:
    $$
        f(x) = \begin{piecewise}
            0 & x=0 \\
            \begin{piecewise}
                1 & x=1 \\
                4 & x=2 \\
                \star & \star
            \end{piecewise} & (x-1)(x-2)=0 \\
            \star & \star
        \end{piecewise}
    $$

    We needn't do this again for our nested piecewise object, but we should instead focus on finding a Newton polynomial form of it:
    $$
        \begin{piecewise}
            1 & x=1 \\
            4 & x=2 \\
            \star & \star
        \end{piecewise} = 1+(x-1)\begin{piecewise}
            \frac{3}{1} & x=2 \\
            \star & \star
        \end{piecewise} = 1+3(x-1)
    $$

    Substituting this into $f(x)$ we have:
    $$
        f(x) = \begin{piecewise}
            0 & x=0 \\
            1+3(x-1) & (x-1)(x-2)=0 \\
            \star & \star
        \end{piecewise}
    $$

    Performing an extraction on the second piece once again we get:
    $$
        f(x) = 1+3(x-1)+(x-1)(x-2)\begin{piecewise}
            \frac{-(1+3(x-1))}{(x-1)(x-2)} & x=0 \\
            \star & \star
        \end{piecewise}
    $$

    We now evaluate the remaining piece, $x=0$ gives $1$, and simplify:
    $$
        f(x) = 1+3(x-1)+(x-1)(x-2)
    $$

    And this gives us a solution for our initial interpolation problem. Also notice that if $f(x)$ is simplified, it gives $x^2$.
\end{example}

\subsubsection{Lagrange polynomials}
Let $f:\mathbb{R}\to\mathbb{R}$ be a function such that $f(x_m)=y_m$ for $m\in\br{0,1,2,\dots,n}$. Then $f$ can be given by a Lagrange polynomial:
$$
    f(x) = \sum_{m=0}^{n}{y_m\prod_{\substack{l=0\\l\neq m}}^{n}{\frac{x-x_l}{x_m-x_l}}}
$$

This form might seem familiar; recall \ref{theorem:iverson_form}. The form of this sum looks similar to the Iverson form of a piecewise object, except that each polynomial term in the sum approximates the Iverson bracket $[x=x_m]$ for some $m$. This is not, in fact, coincidence (and will be covered soon).

\begin{theorem}
    Lagrange polynomials can be formed by separating each piece into its own piecewise object with roots at every other piece, and then interpolating. Setting up our interpolation problem, we have:
    $$
        f(x) = \begin{piecewise}
            y_0 & x=x_0 \\
            y_1 & x=x_1 \\
            \vdots & \vdots \\
            y_n & x=x_n
        \end{piecewise}
    $$

    We can now separate our pieces, similarly to $\ref{theorem:iverson_form}$:
    $$
        f(x) = \begin{piecewise}
            y_0 & x=x_0 \\
            0 & x=x_1 \\
            \vdots & \vdots \\
            0 & x=x_n \\
            \star & \star
        \end{piecewise} + \begin{piecewise}
            0 & x=x_0 \\
            y_1 & x=x_1 \\
            \vdots & \vdots \\
            0 & x=x_n \\
            \star & \star
        \end{piecewise} + \dots + \begin{piecewise}
            0 & x=x_0 \\
            0 & x=x_1 \\
            \vdots & \vdots \\
            y_n & x=x_n \\
            \star & \star
        \end{piecewise}
    $$

    From here, we can interpolate each anonymous piecewise object independently, by extracting each zero piece and simplifying, giving us the Lagrange polynomial as needed.
\end{theorem}

\subsubsection{General solutions}
\label{section:polynomial_general}
As we've seen so far, we've only been producing one solution of a polynomial kind for a set of values for which we wish to interpolate or create an expression for. In general, when we interpolate or perform these extractions, we produce an object in simplified form akin to the following:
$$
    \phi^\star = \mu + [\phi]\cdot r
$$
Where $\mu$ is some solution to the problem, $[\phi]$ is an object which vanishes at the conditions given to us, called a decider (i.e. when those conditions are true), and $r$ is some arbitrary object which generalises our single solution.

\begin{theorem}
    For a polynomial extraction-based anonymous piecewise object problem, we define $\phi^\star$ as the following:
    $$
        \phi^\star = \begin{piecewise}
            \phi & \text{$\phi$ is defined} \\
            \star & \text{$\phi$ is not defined}
        \end{piecewise}
    $$

    There are an infinite number of ways to interpret this, however let us define $\mu$ such that $\mu$ is a `solution' to $\phi^\star$ (that is, it satisfies the definition as above):
    $$
        \mu = \begin{piecewise}
            \phi & \text{$\phi$ is defined} \\
            \star & \text{$\phi$ is not defined}
        \end{piecewise}
    $$

    Therefore, since $\mu=\phi$ for when $\phi$ is defined, we can write $\phi^\star$ as:
    $$
        \phi^\star = \mu + \begin{piecewise}
            0 & \text{$\phi$ is defined} \\
            \star & \text{$\phi$ is not defined}
        \end{piecewise}
    $$

    Let $[\phi]$ be a `decider' object of this problem (that is, it vanishes where $\phi$ is defined). Then we're left with, for some arbitrary $r$:
    $$
        \phi^\star = \mu + [\phi]\cdot r
    $$
\end{theorem}

\begin{theorem}
    The decider of a polynomial interpolation problem with points $\br{(x_0,y_0),(x_1,y_1),\dots,(x_n,y_n)}$ can be given by $\prod_{m=0}^{n}{(x-x_m)}$.

    \begin{proof}
        We denote the corresponding function $F$ as per the following:
        $$
            F(x) = \begin{piecewise}
                y_0 & x=x_0 \\
                y_1 & x=x_1 \\
                \vdots & \vdots \\
                y_n & x=x_n
            \end{piecewise}
        $$

        We can rewrite this function as the following:
        $$
            F(x) = \begin{piecewise}
                y_0 & x=x_0 \\
                y_1 & x=x_1 \\
                \vdots & \vdots \\
                y_n & x=x_n \\
                \star & \star
            \end{piecewise} + \begin{piecewise}
                0 & x=x_0 \\
                0 & x=x_1 \\
                \vdots & \vdots \\
                0 & x=x_n \\
                \star & \star
            \end{piecewise}
        $$

        Extracting the pieces from the second piecewise function, we have:
        $$
            F(x) = \begin{piecewise}
                y_0 & x=x_0 \\
                y_1 & x=x_1 \\
                \vdots & \vdots \\
                y_n & x=x_n \\
                \star & \star
            \end{piecewise} + \begin{piecewise}
                \star & \star
            \end{piecewise}\cdot\prod_{m=0}^{n}{(x-x_m)}
        $$

        Noting that the first piecewise object can be considered a solution, $f$, to the original problem, and the latter piecewise object is arbitrary, e.g. $r$, we have:
        $$
            F(x) = f(x)+r(x)\cdot\prod_{m=0}^{n}{(x-x_m)}
        $$

        Per our definition, we have $\prod_{m=0}^{n}{(x-x_m)}$ as our decider.
    \end{proof}
\end{theorem}

\begin{theorem}
    Let us define a polynomial interpolation problem with the points $\br{(x_0,y_0),\dots,(x_n,y_n)}$.

    Then we have that the function $F(x)=f(x)+r(x)\prod_{m=0}^{n}{(x-x_m)}$ describes all possible solutions to this problem, where $r(x)$ is an arbitrary polynomial.

    \begin{proof}
        While not relevant to the proof, note that the function $F(x)=f(x)+r(x)\prod_{m=0}^{n}{(x-x_m)}$ comes from our general solution form for $\phi^\star$, where the decider of this function is $\prod_{m=0}^{n}{(x-x_m)}$.

        Let us denote the function $P(x)=F(x)-f(x)$

        Since $P(x_m)=0$ by definition for $m\in\br{0,1,\dots,n}$, we have that each $(x-x_m)$ is a root of $P(x)$ by the factor theorem. Therefore, the following is a polynomial:
        $$
            Q(x)=\frac{P(x)}{\prod_{m=0}^{n}{(x-x_m)}}\implies P(x)=Q(x)\prod_{m=0}^{n}{(x-x_m)}
        $$

        Using the definition of $P(x)$ we therefore have that:
        $$
            F(x)=f(x)+Q(x)\prod_{m=0}^{n}{(x-x_m)}
        $$

        Noting that $Q(x)$ is still some arbitrary polynomial as given.
    \end{proof}
\end{theorem}

\begin{theorem}
    A polynomial of degree $n$ can be minimally interpolated using $n+1$ distinct points, and such polynomials are unique.

    \begin{proof}
        Convince yourself that a Lagrange or Newton polynomial as derived above with $n+1$ points produces a polynomial of degree $n$ (without a general solution). Since we've derived these formulas piecewise, it is straightforward to do so: count the number of extractions, corresponding to the number of roots.

        In any case, suppose that two polynomials $p(x)$ and $q(x)$ minimally interpolate $n+1$ points (of degree $n$). Let us define a polynomial $r(x)=p(x)-q(x)$. For each point being interpolated we have $r(x)=0$; namely, $r(x)$ has $n+1$ distinct zeroes. However, since $\deg{r(x)}\leq n$, the fundamental theorem of algebra states that $r(x)$ can have at most $n$ zeroes unless $r(x)=0$. It therefore follows that $p(x)=q(x)$.
    \end{proof}
\end{theorem}

\subsection{Non-trivial piece conditions}
As per our regular piecewise properties, we can perform substitutions on piece conditions of a piecewise object. This is no different for anonymous piecewise objects, or even just piecewise objects when performing extractions as we have so far. With this being said, there are caveats.

As we've observed with Examples \ref{example:max_further} and \ref{example:heaviside_further}, particularly with $\abs{x}$, overlapping conditions produce discontinuities and ambiguities. Sometimes, like in the former example, the resultant discontinuities are removable, but most times (such as in the latter example), they are not. Ultimately, operations on piecewise objects are only `nice' when these piecewise objects are well defined; it is important, therefore, to keep this in mind when performing piece condition substitution.

\begin{theorem}
    Given some piecewise object $\phi$ as defined below:
    $$
        \phi=\pwobj{\phi_i}{C_i}{i\in I}
    $$

    If there exists a substitution $C_i\rightarrow x_i=y_i$, then not only can we write $\phi$ as the following:
    $$
        \phi=\pwobj{\phi_i}{x_i=y_i}{i\in I}
    $$

    But we can also perform our extraction techniques on the piecewise object. For example, $x\geq y\iff \abs{x-y}=x-y$. This has previously been demonstrated, but we now make explicit this as a general technique.
\end{theorem}

This also means that when we interpolate a set of points, we needn't produce a polynomial in the variable being given.

\begin{example}
    We wish to produce a function, not strictly a polynomial, $f(x)$ such that $f(0)=0$, $f(1)=1$ and $f(\frac{\pi}{2})=1$.

    We can do so, for example, per the following:
    $$
        f(x) = \begin{piecewise}
            0 & x=0 \\
            1 & x=1 \\
            1 & x=\frac{\pi}{2} \\
            \star & \star
        \end{piecewise}
    $$

    Let us substitute the condition $x=\frac{\pi}{2}\implies \sin\paren{x}=1$, noting that $f(x)$ remains well-defined:
    $$
        f(x) = \begin{piecewise}
            0 & x=0 \\
            1 & x=1 \\
            1 & \sin\paren{x}=1 \\
            \star & \star
        \end{piecewise}
    $$

    From here, we'll subtract $1$ from the piecewise object so that we can extract the latter two pieces:
    $$
        f(x) = 1+(x-1)(\sin\paren{x}-1)\begin{piecewise}
            \frac{-1}{(x-1)(\sin\paren{x}-1)} & x=0 \\
            \star & \star
        \end{piecewise}
    $$

    This yields $f(x)=1-(x-1)(\sin\paren{x}-1)=x-x\sin\paren{x}+\sin\paren{x}$.
\end{example}
\subsection{Multiple conditions}
Up until this point we've been dealing with piecewise objects whose pieces only have one condition (or objects whose conditions can't particularly be simplified by nesting); dealing with multiple conditions is far more difficult. With this being said, there are multiple strategies which we can employ to deal with various situations, including formulations for logical `or' and logical `and'.

We begin by dealing with the cases wherein each condition is an equality of some kind.
\begin{theorem}
    Suppose $x\in S$ where $S$ is some nicely-defined set. If the following expression exists, we can describe $x$ in a single equality using the following:
    $$
        x\in S\leftrightarrow \prod_{s\in S}{(x-s)}=0
    $$
    This is sometimes known as the null factor law.

    \begin{proof}
        \label{theorem:set_variable}
        Let $x\in S$ denote the following piecewise object which is not well-defined:
        $$
            x = \pwobj{s}{x=s}{s\in S}
        $$
        That is, $x$ is equal to all elements in the set it is itself an element of. Using this, we have that:
        $$
            \pwobj{x-s}{x=s}{s\in S}=0
        $$

        Since $x=s$ in each piece, we have $x-s=0$ and so simplifying each piece we get:
        $$
            \pwobj{0}{x=s}{s\in S}=0
        $$

        Extracting each piece yields:
        $$
            \prod_{s\in S}{(x-s)}=0
        $$
    \end{proof}
\end{theorem}
\begin{theorem}
    \label{theorem:logical_or_equality}
    The following is a logical `or' simplification for equalities.

    Suppose we have a condition $A\leftrightarrow \displaystyle\bigvee_{i\in I}{A_i}$ such that $A_i\leftrightarrow(x_i=y_i)$. Then we have that $A$ can be given by the following:
    $$
        A\leftrightarrow\prod_{i\in I}{(x_i-y_i)}=0
    $$

    \begin{proof}
        For all $i\in I$ we have that $x_i=y_i\implies x_i-y_i=0$. It therefore stands to reason that $0\in\br*{x_i-y_i\mid i\in I}\leftrightarrow A$.

        By \ref{theorem:set_variable} we have that $A\leftrightarrow \prod_{i\in I}{(x_i-y_i)}=0$.
    \end{proof}
\end{theorem}
\begin{theorem}
    The following is a logical `or' simplification for equalities.

    Suppose we have a condition $A\leftrightarrow \displaystyle\bigwedge_{i\in I}{A_i}$ such that $A_i\leftrightarrow(x_i=y_i)$. Then we have that $A$ can be given by the following:
    $$
        A\leftrightarrow\sum_{i\in I}{p_i(x_i-y_i)}=0
    $$
    Where $p_i$ is some function such that $p_i(x)=0\iff x=0$, and $p_i(x)\geq 0$.

    \begin{proof}
        Let $\vec{v}$ be a vector with components $p_i(x_i-y_i)$ for $i\in I$. Consider that the following are equivalent statements:
        \begin{align*}
            &\vec{v} = \mathbf{0} \\
            &\bigwedge_{i\in I}{\paren*{p_i(x_i-y_i) = 0}}
        \end{align*}

        We can take the norm of the latter statement, giving us:
        $$
            \sum_{i\in I}{\abs{p_i(x_i-y_i)}}=0
        $$

        Since $p_i\geq 0$, we have that:
        $$
            \sum_{i\in I}{p_i(x_i-y_i)}=0
        $$

        To prove that this is an iff relationship, let $j\in I$. Using the above statement, we have:
        $$
            p_j(x_j-y_j)=-\sum_{\substack{i\in I\\i\neq j}}{p_i(x_i-y_i)}\leq 0
        $$

        But since $p_j(x_j-y_j)\geq 0$, we have that $p_j(x_j-y_j)=0$ for all $j\in I$.
    \end{proof}
\end{theorem}

\subsection{Further extraction}
\textit{Note, this is very much a section I wish to work on further and requires more knowledge in abstract algebra than I currently possess. As it stands I'm missing terminology, concepts and whatever else.}

So far, we've been doing what I call in these notes `polynomial extraction'. Really, this is a process of two major steps: addition and multiplication, and then their respective inverses. That is, for each piece, we've vanished the piece value by adding (and subtracting), and then by reducing by multiplying the corresponding root (and dividing in each other piece), given by the piece condition. As it turns out, this process is not restricted to these operations.

Let $S$ be a set such that we have binary operations $f:S^2\to S$, $g:S^2\to S$, $f^{-1}:S^2\to S$, $g^{-1}:S^2\to S$ with the properties (for all $x,y\in S$):
\begin{align*}
    f(f^{-1}(x,y),y) &= x \\
    \exists a\in S\ni f^{-1}(x,x) &= a \\
    f(a,x) &= x & \textit{(by the first two properties)} \\
    g(g^{-1}(x,y),y) &= x \\
    \exists b\in S\ni g(x,b) &= a
\end{align*}
Where $a,b$ are special elements of $S$ (and $\ni$ refers to `such that').

Then, our polynomial extraction processes uses the operations: $f(x,y)=x+y$, $f^{-1}(x,y)=x-y$, $g(x,y)=x\cdot y$ and $g^{-1}(x,y)=\frac{x}{y}$.

There is, in fact, another: exponential extraction. That is we define $f(x,y)=x\cdot y$, $f^{-1}(x,y)=\frac{x}{y}$, $g(x,y)=x^y$ and $g^{-1}(x,y)=x^\frac{1}{y}$.

A few notes to be made here: firstly, we don't strictly have to define our operations in such a way (there are other ways to do so, such as changing the order of inverses and whatnot). Furthermore, our substitution property holds regardless of extraction method, where appropriate or applicable. In any case, we can actually derive these properties from an `ideal' process such as with the polynomial or exponential extractions.

\begin{example}
    Let us motivate these properties using an interpolation problem for points $(0,u)$ and $(1,v)$.

    \par\textbf{Polynomial interpolation.}
    \begin{align*}
        F(x) &= \begin{piecewise}
            u & x=0 \\
            v & x=1 \\
            \star & \star
        \end{piecewise} \\
        &= \begin{piecewise}
            0 & x=0 \\
            v-u & x=1 \\
            \star & \star
        \end{piecewise}+u \\
        &= \begin{piecewise}
            \frac{v-u}{x} & x=1
        \end{piecewise}\cdot x+u \\
        &= (v-u)x+u
    \end{align*}

    \par\textbf{Generalised interpolation.}
    Let $g:S\to S$ be a function such that $G(0)=b$.
    \begin{align*}
        F(x) &= \begin{piecewise}
            u & x=0 \\
            v & x=1 \\
            \star & \star
        \end{piecewise} \\
        &= f\paren*{\begin{piecewise}
            f^{-1}(u,u) & x=0 \\
            f^{-1}(v,u) & x=1 \\
            \star & \star
        \end{piecewise},u} \\
        &= f\paren*{\begin{piecewise}
            a & G(x)=b \\
            f^{-1}(v,u) & G(x)=G(1) \\
            \star & \star
        \end{piecewise},u} \\
        &= f\paren*{g\paren*{\begin{piecewise}
            g^{-1}\paren*{f^{-1}(v,u),G(x)} & G(x)=G(1) \\
            \star & \star
        \end{piecewise},G(x)},u} \\
        &= f\paren*{g\paren*{g^{-1}\paren*{f^{-1}(v,u),G(1)},G(x)},u}
    \end{align*}
\end{example}

\begin{theorem}
    An anonymous piecewise object is given by the following:
    $$
        \phi^\star = \begin{piecewise}
            \phi & \text{$\phi$ is defined} \\
            \star & \text{$\phi$ is not defined}
        \end{piecewise}
    $$

    Let $\mu$ be a solution to $\phi^\star$ given our operations. Then we can write $\phi^\star$ as:
    $$
        \phi^\star = f\paren*{g\paren*{r,[\phi]},\mu}
    $$
    Where $r$ is some arbitrary object, $[\phi]$ is the decider object (i.e. $[\phi]=b$ when $\phi$ is defined) and $\mu$ is a solution to $\phi^\star$.
\end{theorem}

\newpage