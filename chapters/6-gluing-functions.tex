\section{Gluing Functions}
Until now, we've discussed piecewise concepts in the more abstract sense, with explicit examples permeating the paradigm behind them as previously discussed. From here, we'll begin introducing new concepts with a heavier focus on explicit constructions. This begins with our gluing functions.

\subsection{Continuous gluing function}
Let us define intervals $D_1,D_2,\dots,D_n$ such that for $m\in\br{1,2,\dots,n-1}$ we have $\max\paren{D_m}=\min\paren{D_{m+1}}$. Then we define all respective functions to be glued $f_m:D_m\to\mathbb{R}$. To `glue' such functions together sequentially in a function $F(x)$ we write:
$$
    F(x)=\sum_{m=1}^{n}{f_m\paren*{\ell_{a_m}^{b_m}\paren{x}}}-\sum_{m=2}^{n}{f_m\paren*{a_m}}
$$
Where $a_m=\min\paren{D_m}$ if it exists, or $-\infty$ otherwise. Likewise $b_m=\max\paren{D_m}$ if it exists, or $\infty$ otherwise. This also uses the clamping function, as given in \ref{section:clamping_function}.

\begin{proof}
    This derivation actually follows as an extension of Example \ref{example:gluing_basic}. We begin by letting $F:\mathbb{R}\to\mathbb{R}$ be a function such that:
    $$
        F(x) = \pwobj{f_m\paren{x}}{x\in D_m}{m\in\br{1,2,\dots,n}}
    $$

    Let us define the following:
    \begin{align*}
        a_m &= \begin{piecewise}
            \min\paren{D_m} & \text{if $\min\paren{D_m}$ exists} \\
            -\infty & \text{otherwise}
        \end{piecewise} \\
        b_m &= \begin{piecewise}
            \max\paren{D_m} & \text{if $\max\paren{D_m}$ exists} \\
            \infty & \text{otherwise}
        \end{piecewise}
    \end{align*}

    Therefore $x\in D_m\iff \ell_{a_m}^{b_m}\paren{x}=x$ (note, per the definition of the clamping function, this skips the nuances of having closed/open intervals, etc.). We can accordingly rewrite $F(x)$:
    $$
        F(x) = \pwobj{f_m\paren{x}}{\ell_{a_m}^{b_m}\paren{x}=x}{m\in\br{1,2,\dots,n}}
    $$

    Substituting the respective conditions where $x=\ell_{a_m}^{b_m}\paren{x}$, we have that:
    $$
        F(x) = \pwobj{f_m\paren*{\ell_{a_m}^{b_m}\paren{x}}}{\ell_{a_m}^{b_m}\paren{x}=x}{m\in\br{1,2,\dots,n}}
    $$

    And then we can `subtract' each piece out:
    $$
        F(x) = \sum_{l=1}^{n}{f_l\paren*{\ell_{a_l}^{b_l}\paren{x}}}+\pwobj{-\sum_{\substack{l=1\\l\neq m}}^{n}{f_l\paren*{\ell_{a_l}^{b_l}\paren{x}}}}{\ell_{a_m}^{b_m}\paren{x}=x}{m\in\br{1,2,\dots,n}}
    $$

    Focusing on the sum in the piecewise function, we have:
    $$
        \sum_{\substack{l=1\\l\neq m}}^{n}{f_l\paren*{\ell_{a_l}^{b_l}\paren{x}}} = \sum_{l=1}^{m-1}{f_l\paren*{\ell_{a_l}^{b_l}\paren{x}}}+\sum_{l=m+1}^{n}{f_l\paren*{\ell_{a_l}^{b_l}\paren{x}}}
    $$

    Since $x\in D_m$ we know that for the first partial sum, we have $f_l(b_l)$ for each term. Likewise in the second sum, we have $f_l(a_l)$ for each term. Noting that $f_{l+1}(a_{l+1})=f_l(b_l)$ by our continuity property, we can rewrite the sums respectively:
    $$
        \sum_{\substack{l=1\\l\neq m}}^{n}{f_l\paren*{\ell_{a_l}^{b_l}\paren{x}}} = \sum_{l=1}^{m-1}{f_{l+1}\paren*{a_{l+1}}}+\sum_{l=m+1}^{n}{f_l\paren*{a_l}}
    $$

    In the first sum, we write $l\to l-1$ and so we have, for each $m\in\br{1,2,\dots,n}$:
    $$
        \sum_{\substack{l=1\\l\neq m}}^{n}{f_l\paren*{\ell_{a_l}^{b_l}\paren{x}}} = \sum_{l=2}^{n}{f_{l}\paren*{a_{l}}}
    $$

    Therefore, $F(x)$ is given by:
    $$
        F(x) = \sum_{l=1}^{n}{f_l\paren*{\ell_{a_l}^{b_l}\paren{x}}}+\pwobj{-\sum_{l=2}^{n}{f_{l}\paren*{a_{l}}}}{\ell_{a_m}^{b_m}\paren{x}=x}{m\in\br{1,2,\dots,n}}
    $$

    Since each piece is identical, we can reduce our piecewise object accordingly:
    $$
        F(x) = \sum_{l=1}^{n}{f_l\paren*{\ell_{a_l}^{b_l}\paren{x}}}-\sum_{l=2}^{n}{f_{l}\paren*{a_{l}}}
    $$
\end{proof}

As a corollary, if all of $f_1,f_2,\dots,f_n$ are elementary functions, then $F(x)$ is itself an elementary function as $\ell_{a_m}^{b_m}\paren{x}$ is, also.

The derivation of the gluing function above doesn't actually do the function, nor the piecewise techniques involved, much justice. With this being said, it was derived with full generality, and other such derivations may need to deal with nuances more directly on a case-by-case basis (such as when $\min\paren{D_1}$ doesn't exist). Following this, I leave it as an exercise to the reader to derive this formula using, perhaps, induction, where each interval is finite (bounded above/below). This article \href{https://piecewise.org/exploration/gluing-functions}{Gluing Functions} can be used for inspiration.

\subsection{Generalised gluing function in one dimension}
Recall that in \ref{theorem:iverson_form} we explored the Iverson bracket form of a piecewise object when each condition was `disjoint'. That is, the following are equivalent:
\begin{align*}
    \phi &= \pwobj{\varphi_i}{C_i}{i\in I} \\
    \phi &= \sum_{i\in I}{\varphi_i[C_i]}
\end{align*}

We can, in fact, reduce this form down for the purposes of our gluing function. Let us denote the sets $D_1,D_2,\dots,D_n$ such that for all $i,j\in\br{1,2,\dots,n},\ i\neq j$, we have $D_i\cap D_j=\varnothing$ (that is, every set is disjoint from one another). Then let us define functions $f_m:D_m\to\mathbb{R}$ for $m\in\br{1,2,\dots,n}$, and $F:\mathbb{R}\to\mathbb{R}$ such that:
$$
    F(x) = \sum_{m=1}^{n}{f_m\paren{x}[x\in D_m]}
$$

More explicitly, what we're summing is the following:
$$
    F(x) = \begin{piecewise}
        f_1\paren{x} & x\in D_1 \\
        0 & x\not\in D_1
    \end{piecewise} + \begin{piecewise}
        f_2\paren{x} & x\in D_2 \\
        0 & x\not\in D_2
    \end{piecewise} + \dots + \begin{piecewise}
        f_n\paren{x} & x\in D_n \\
        0 & x\not\in D_n
    \end{piecewise}
$$

In fact, we can reach our continuous gluing function formula if what we have is continuous, per our usual techniques (although since our sets are disjoint, we resort to using $\sup$ and $\inf$ rather than $\max$ and $\min$, respectively).

\subsection{The triangle wave}
Let us consider the function $f(x)=\arcsin\paren{\sin\paren{x}}$ (a similar function is $g(x)=\arccos\paren{\cos{x}}$ though we won't derive it here). This function, is, in fact, piecewise, and can be written as:
$$
    f(x) = \frac{\pi}{2}-\abs{x-\frac{\pi}{2}\paren*{1+4\floor*{\frac{x}{2\pi}+\frac{1}{4}}}}
$$

\begin{proof}
    We use the techniques established in \ref{section:function_composition} to derive the above formula. Using the definition of $f(x)$ we have that:
    \begin{align*}
        f(x)=\arcsin\paren{\sin\paren{x}} &\implies \sin\paren{f(x)}=\sin\paren{x} \\
        &\implies \sin\paren{f(x)}-\sin\paren{x}=0
    \end{align*}

    Recall the identity $\sin\paren{a}-\sin\paren{b}=2\sin\paren*{\frac{a-b}{2}}\cos\paren*{\frac{a+b}{2}}$, and so the above is equivalent to:
    $$
        2\sin\paren*{\frac{f(x)-x}{2}}\cos\paren*{\frac{f(x)+x}{2}}=0
    $$

    The solutions to this equation are as follows\footnotemark:
    \begin{align}
        \sin\paren*{\frac{f(x)-x}{2}}=0 &\implies f(x)=2m\pi+x & m\in\mathbb{Z} \\
        \cos\paren*{\frac{f(x)+x}{2}}=0 &\implies f(x)=(2n+1)\pi-x & n\in\mathbb{Z}
    \end{align}

    \footnotetext{Note, if you're not familiar with how these identities and solutions have come about, while not necessary, it is a fun exercise to recognise how they have. I found that abusing Euler's formula to derive the formulas is pretty fun.}

    From here, we use the fact that $f(x)\in\brak*{-\frac{\pi}{2},\frac{\pi}{2}}$ to establish respective domains for our solutions:
    \begin{align}
        2m\pi+x\in\brak*{-\frac{\pi}{2},\frac{\pi}{2}} &\implies x\in\brak*{-\frac{\pi}{2}(4m+1),-\frac{\pi}{2}(4m-1)} \\
        (2n+1)\pi-x\in\brak*{-\frac{\pi}{2},\frac{\pi}{2}} &\implies x\in\brak*{\frac{\pi}{2}(4n+1),\frac{\pi}{2}(4n+3)}
    \end{align}

    From here, we write $f(x)$ in piecewise form (noting that it is a continuous function as both $\arcsin$, $\sin$ are elementary, continuous functions on $\mathbb{R}$), and also noting that we can rewrite $m\to-n$:
    \begin{align*}
        f(x) &= \pwobj{-2n\pi+x}{x\in\brak*{\frac{\pi}{2}(4n-1),\frac{\pi}{2}(4n+1)}}{n\in\mathbb{Z}} \\
             &\cup \pwobj{(2n+1)\pi-x}{x\in\brak*{\frac{\pi}{2}(4n+1),\frac{\pi}{2}(4n+3)}}{n\in\mathbb{Z}}
    \end{align*}
\end{proof}

\newpage