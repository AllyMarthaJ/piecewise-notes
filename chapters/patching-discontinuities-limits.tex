% yay we revisit non elementary functions
% step functions for example, approximations
\chapter{Discontinuous Piecewise Functions}
\section{Limit Method}
We've already explored some elementary functions which can approximate piecewise functions (continuous or not). Some functions, however aren't covered, and, moreover, there is no systematic way to derive such approximations except via substitution using existing functions. This is what we cover here, with some extra steps.

Let $f$ be some discontinuous piecewise function. Then we shall define $f$ using a limit. For all real $x$, let us define the following:
$$
    \lim_{\varepsilon\to0}{F(x,\abs{\varepsilon})}=f(x)
$$

That is, there exists some function $F$ which is exactly $f$ everywhere except around its discontinuities. In order to formulate such a function, we have the properties below. For some small $\varepsilon>0$ we have the following:

\begin{enumerate}
    \item If $\lim_{x\to a^+}{f(x)}=b$ then $f(a+\varepsilon)\approx b$.
    \item If $\lim_{x\to a^-}{f(x)}=b$ then $f(a-\varepsilon)\approx b$.
    \item If $\lim_{x\to a}{f(x)}=b$ then $f(a)\approx b$ (if $f(a)$ exists).
\end{enumerate}

This should look familiar; we're given a set of individual points, which means can interpolate these points where appropriately. Furthermore, since we interpolate only in the region of the discontinuity, we can then apply our gluing formula to the rest of the domain, as derived in Section \ref{section:gluing}.

\begin{example}
    Let us find a function $F$ as per the above which approximates $\sgn{x}$. Recall that $\sgn{x}$ is defined as the following:
    $$
        \sgn{x} = \begin{piecewise}
            1 & x > 0 \\
            0 & x = 0 \\
            -1 & x < 0
        \end{piecewise}
    $$

    For some small $x\in [-\varepsilon, \varepsilon]$ for $\varepsilon >0$, therefore, we have that:
    $$
        F(x,\varepsilon) = \begin{piecewise}
            1 & x = \varepsilon \\
            0 & x = 0 \\
            -1 & x = -\varepsilon \\
            \star & \star
        \end{piecewise}
    $$

    If we view this as an interpolation problem, this gives $F(x,\varepsilon)=\frac{x}{\varepsilon}$. Combining this with the remainder of the domain, $\mathbb{R}$, we have that:
    $$
        F(x,\varepsilon) = \begin{piecewise}
            1 & x\geq\varepsilon \\
            \frac{x}{\varepsilon} & -\varepsilon\leq x\leq \varepsilon \\
            -1 & x\leq-\varepsilon
        \end{piecewise}
    $$

    Applying our gluing formula, we get:
    $$
        F(x,\varepsilon) = \paren*{1 + \frac{\ell_{-\varepsilon}^{\varepsilon}(x)}{\varepsilon} + (-1)} - \paren*{1 + (-1)}
    $$

    Fully simplifying using the properties of the clamping function given in \ref{section:clamping_function}, this leaves:
    $$
        F(x,\varepsilon) = \ell_{-1}^{1}\paren*{\frac{x}{\varepsilon}}
    $$

    It therefore stands to reason that:
    $$
        \lim_{\varepsilon\to 0}{\ell_{-1}^{1}\paren*{\frac{x}{\abs{\varepsilon}}}}=\sgn{x}
    $$
\end{example}

\begin{example}
    \label{example:floor_sum}
    Recall that the floor function is defined as the following:
    $$
        \floor{x} = \pwobj{n}{x\in[n,n+1)]}{n\in\mathbb{Z}}
    $$

    Using our patching method, we can also define $\floor{x}$ using the following:
    $$
        \floor{x} = \lim_{N\to\infty}{\paren*{N+\sum_{n=-N}^{N}{\lim_{\varepsilon\to0}{\ell_{-1}^{0}\paren*{\frac{x-n}{\abs{\varepsilon}}}}}}}
    $$

    \begin{proof}
        The floor function is handled differently to other functions in that it has infinitely many discontinuities. So instead of trying to handle it for each integer section, let us consider, for all $n\in\mathbb{Z}$ the interval $x\in\brak*{n-\frac{1}{2},n+\frac{1}{2}}$:
        $$
            \floor{x} = \begin{piecewise}
                n-1 & x<n \\
                n & x\geq n
            \end{piecewise}
        $$

        We can therefore construct $F_n(x,\varepsilon)$ as follows:
        $$
            F_n(x,\varepsilon) = \begin{piecewise}
                n-1 & x=n-\varepsilon \\
                n & x=n \\
                \star & \star
            \end{piecewise}
        $$

        Which gives:
        $$
            F_n(x,\varepsilon) = n+\frac{x-n}{\varepsilon}
        $$

        Gluing, we have:
        $$
            F_n(x,\varepsilon)= \ell_{n-1}^{n}\paren*{n+\frac{x-n}{\varepsilon}}
        $$

        We therefore have that our `total' function is given by:
        $$
            \floor{x} = \pwobj{\lim_{\varepsilon\to0}{\ell_{n-1}^{n}\paren*{n+\frac{x-n}{\abs{\varepsilon}}}}}{x\in\brak*{n-\frac{1}{2},n+\frac{1}{2}}}{n\in\mathbb{Z}}
        $$

        Now that this function is continuous at the endpoints of each interval, we can glue using partial sums:
        $$
            f_N(x) = \sum_{n=-N}^{N}{\lim_{\varepsilon\to0}{\ell_{n-1}^{n}\paren*{n+\frac{\ell_{n-\frac{1}{2}}^{n+\frac{1}{2}}(x)-n}{\abs{\varepsilon}}}}}-\sum_{n=1-N}^{N}{\paren*{n-1}}
        $$

        Note that the latter sum is equal to $-N$ (left as an exercise to the reader). Simplifying the inside of the former sum (and doing the same with the $n$ term which sums to $0$), however, we get:
        $$
            f_N(x) = N+\sum_{n=-N}^{N}{\lim_{\varepsilon\to0}{\ell_{-1}^{0}\paren*{\ell_{-\frac{1}{2\abs{\varepsilon}}}^{\frac{1}{2\abs{\varepsilon}}}\paren*{\frac{x-n}{\abs{\varepsilon}}}}}}
        $$

        Since $\varepsilon\to 0$ the inner clamping function is unbounded (i.e. $\ell_{-\infty}^{\infty}(x)=x$), so we have:
        \begin{align*}
            f_N(x) &= N+\sum_{n=-N}^{N}{\lim_{\varepsilon\to0}{\ell_{-1}^{0}\paren*{\frac{x-n}{\abs{\varepsilon}}}}}\\
            \floor{x} &= \lim_{N\to\infty}{f_N(x)}
        \end{align*}
    \end{proof}
\end{example}

\begin{theorem}
    Using the floor function per Example \ref{example:floor_sum}, we can restate our proposition as the following:

    For all $\abs{x}\leq N\in\mathbb{Z}^+$ we have that:
    $$
        \floor{x} = N+\sum_{n=-N}^{N}{\lim_{\varepsilon\to0}{\ell_{-1}^{0}\paren*{\frac{x-n}{\abs{\varepsilon}}}}}
    $$

    \begin{proof}
        For all $\abs{x}\leq N$, we know that $\exists m\in\br{-N,\dots,N-1}$ such that $m\leq x< m+1$. Then:

        $$
            \floor{x} = N+\sum_{n=-N}^{m}{\lim_{\varepsilon\to0}{\ell_{-1}^{0}\paren*{\frac{x-n}{\abs{\varepsilon}}}}}+\sum_{n=m+1}^{N}{\lim_{\varepsilon\to0}{\ell_{-1}^{0}\paren*{\frac{x-n}{\abs{\varepsilon}}}}}
        $$

        Then for all $n\leq m$ we have $x\geq n\implies\frac{x-n}{\abs{\varepsilon}}\geq 0$. Using the definition of the clamping function, therefore, we have that the summand of the first sum is 0 (after each limit is taken). Likewise for the second sum, we have $m\leq n$ which means $\frac{x-n}{\abs{\varepsilon}}\leq 0$. Taking the limits, this means we have $-1$ for all terms, giving us:
        $$
            \floor{x} = N+\sum_{n=m+1}^{N}{(-1)}
        $$

        This gives $\floor{x}=N-(N-m)=m$.
    \end{proof}
\end{theorem}

\section{Sign-related substitutions}
Recall that we use the substitution $\max\br{x,a}=x$ for our continuous function gluing. This has the advantage that we invoke functions within their respective domains only (where the endpoints of the domain are defined). Unfortunately, this method doesn't work if the endpoints are undefined, but the function doesn't diverge. Such an example is $\exp\paren*{-\frac{1}{x^2}}$.

Taking inspiration from our $\max$ function substitution, for problems which require proper domain handling, we can perform a $\sgn{x}$ related substitution, using an auxillary function or functions which reduce the problem.

\begin{example}
    We want to write the function $f:\mathbb{R}\to\mathbb{R}$ in terms of $\sgn{x}$ so that it is defined everywhere:
    $$
        f(x) = \begin{piecewise}
            \sqrt{\frac{1}{x}} & x>0 \\
            0 & x\leq 0
        \end{piecewise}
    $$

    In order to do this, let us create an auxillary function, $A:\mathbb{R}\to\mathbb{R}$, to allow us appropriately create a substitution for $x$ when $x>0$. Suppose $a\in\mathbb{R}^+$, then such a function could be:
    $$
        A(x) = \begin{piecewise}
            x & x > 0 \\
            a & x\leq 0
        \end{piecewise}
    $$

    Then $x>0\iff A(x)=x$ and $x\leq 0\iff A(x)=a$ (which we will ignore). Returning to $f$, we get:
    $$
        f(x) = \begin{piecewise}
            \sqrt{\frac{1}{A(x)}} & A(x)=x \\
            0 & x\leq 0
        \end{piecewise} = \sqrt{\frac{1}{A(x)}}+\begin{piecewise}
            0 & x > 0 \\
            0-\sqrt{\frac{1}{A(x)}} & x\leq 0
        \end{piecewise}
    $$

    When simplified, we get:
    $$
        f(x) = \sqrt{\frac{1}{A(x)}}-\sqrt{\frac{1}{a}}+\begin{piecewise}
            \sqrt{\frac{1}{a}} & x > 0 \\
            0 & x\leq 0
        \end{piecewise}
    $$

    This gives motivation to set $a=1$. We will further simplify using techniques explored previously, giving us:
    $$
        f(x)=\sqrt{\frac{1}{A(x)}}+\frac{1}{2}\sgn{x}(\sgn{x}+1)-1
    $$

    Since $A(x)=a+\frac{x-a}{2}\sgn{x}(\sgn{x}+1)$:
    $$
        f(x)=\sqrt{\frac{1}{a+\frac{x-a}{2}\sgn{x}(\sgn{x}+1)}}+\frac{1}{2}\sgn{x}(\sgn{x}+1)-1
    $$
\end{example}

\begin{theorem}
    Let $D\subseteq\mathbb{R}$, $a\in D$ and $f:D\setminus\br{a}\to\mathbb{R}$ such that $f(a)=s$. Then for some $r\in D$ the function $F:D\to\mathbb{R}$ patches $f$:
    $$
        F(x) = f(x+(x-r)(\sgn{x-a}^2-1))+(f(r)-s)(\sgn{x-a}^2-1)
    $$

    \begin{proof}
        Let $F:D\to\mathbb{R}$ be a function such that:
        $$
            F(x) = \begin{piecewise}
                f(x) & x\neq a\\
                s & x=a
            \end{piecewise}
        $$

        We create a patching (auxillary) function $P:D\to\mathbb{R}$ such that
        $$
            P(x,a) = \begin{piecewise}
                r & x=a \\
                x & x\neq a
            \end{piecewise} = x + (x-r)(\sgn{x-a}^2-1)
        $$

        Rewriting $F$:
        $$
            F(x) = \begin{piecewise}
                f(x) & P(x,a)=x \\
                s & P(x,a)=r
            \end{piecewise} = f(P(x,a)) + (f(r)-s)(\sgn{x-a}^2-1)
        $$

        Simplifying yields the result as desired.
    \end{proof}
\end{theorem}

\newpage