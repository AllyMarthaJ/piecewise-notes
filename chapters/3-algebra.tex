\section{Algebra on Piecewise Objects}
\subsection{Piece association}
Operations on piecewise objects are fairly straightforward and behave as any other object in the context you're working in (whether this be with the real numbers, complex numbers, certain algebras, spaces, and so on). 

What this section aims to do is not to teach you directly how to perform exactly the operations you want to on each piecewise object, but instead provide a basis for which you can base your ideas: no piece of a piecewise object is independent of another. Equivalently, each piece is dependent on each other piece. There is an intuitive reason for this: you are evaluating an object not conditionally, but in full generality; each piece exists, in some sense, `simultaneously'.

It is here we might recognise that this idea could fairly easily lead into combinatorics; an area of maths which deals heavily with the enumeration and construction of such objects.
\subsection{Equality property}
\begin{theorem}
    Consider the following piecewise object:

    $$
        \phi=\pwobj{\phi_i}{C_i}{i\in I}
    $$

    Then if for all $i,j\in I$ we have that $\phi_i=\phi_j$ (that is, all piece values are equal for when $\phi$ is defined), then $\phi=\phi_i$ for any $i\in I$.

    For example, consider the function $f:(-\infty,0]\to\mathbb{R}$:

    $$
        f(x) = \begin{piecewise}
            0 & x \geq 0 \\
            0 & x < 0
        \end{piecewise}
    $$

    We have that $f(x)=0$. This is because everywhere in $\dom{f}$, or both when $x\geq 0$ and $x < 0$, we have that $f(x)=0$.
\end{theorem}

\subsection{Piece and condition equivalence}
Arguably one of the most fundamental property of piecewise objects is the relationship between each piece's conditions and its respective value. That is, the values for which a piecewise object takes is dependent on their respective conditions, and so can be treated explicitly as such.

\begin{theorem}
    Consider the following piecewise function:

    $$
        f(x) = \pwobj{f_i(x)}{C_i}{i\in I}
    $$

    For all $i\in I$, let us define a substitution $x\sim y_i$, where $\sim$ represents equality under the condition $C_i$. Then we might consider writing $f(x)$ as:

    $$
        f(x) = \pwobj{f_i(y_i)}{C_i}{i\in I}
    $$

    And $f(x)$ may still remain a non-constant function, but in this way we've given it another representation. For some $C_i$ we might just have $x\sim x$ (and is effectively no substitution at all). This substitution can also go the other way, i.e. $y_i\sim x$.
\end{theorem}

\begin{theorem}
    Consider the following piecewise object:

    $$
        \phi=\pwobj{\phi_i}{C_i}{i\in I}
    $$

    For all $i\in I$, suppose that $C_i \leftrightarrow D_i$ in the context of $\phi$ (i.e. if a function, within its domain, etc.). Then we can rewrite this object as:

    $$
        \phi=\pwobj{\phi_i}{D_i}{i\in I}
    $$ 

    That is, we've substituted the set of our conditions for another set of conditions (and again, we may not have substituted all of them). For example, we know that $x=5$ when $(x-5)^2\leq 0$, and so these two conditions are interchangeable.
\end{theorem}

This idea becomes far more important when working using other functions, such as $\max$, $\min$ and $\abs{x}$ (which will be covered later), although it can still be used in other contexts. In essence, for certain classes of piecewise functions, we can build off existing functions to construct not only transformations, but completely new functions and representations.
\begin{example}
    \label{example:equality_1}
    As an example, let us consider the function $f:(-6, 6)\to\mathbb{R}$:

    $$
        f(x) = \begin{piecewise}
            x & x > 5 \\
            5 & x = 5 \\
            x & x < 5
        \end{piecewise}
    $$

    Consider the piece for which $x=5$: note that the piece value is equal to $5$. Since $x=5$ we can replace $5$ with $x$ to get:

    $$
        f(x) = \begin{piecewise}
            x & x > 5 \\
            x & x = 5 \\
            x & x < 5
        \end{piecewise}
    $$

    And by the equality property, we know that this function is equivalent to $f(x)=x$.
\end{example}

Finally, depending on the context of the problem, function or object, we can rewrite conditions in some way to make it more straightforward for us to understand. This becomes relevant in the context of function domains.

\begin{example}
    Let us consider the function from Example $\ref{example:equality_1}$.

    The following two representations are equivalent:

    $$
        f(x) = \begin{piecewise}
            x & x > 5 \\
            5 & x = 5 \\
            x & x < 5
        \end{piecewise}
    $$

    $$
        f(x) = \begin{piecewise}
            x & x\in(5,6) \\
            5 & x = 5 \\
            x & x\in(-6,5)
        \end{piecewise}
    $$

    The reason for this is that for all $x\in(-6,5)$ we have that $x<5$, despite the contrary not being always true. The reason the contrary needn't always be true for all $\mathbb{R}$ is because our function $f$ is defined on the interval $(-6,6)$. This same argument can also be applied to $x\in(5,6)\implies x>5$.
\end{example}

A strong note on functions which are not well-defined: Piecewise objects can easily define not well-defined functions. In these cases, care must be taken when working with pieces in general, values or conditions. 

\begin{theorem}
    A piecewise function is only well-defined if all of its piece values denote well-defined functions, and in places where two conditions in separate pieces are true, their respective values must be equal.

    Formally, given the piecewise object $\phi=\pwobj{\phi_i}{C_i}{i\in I}$ if there exists $i\neq j\in I$ such that $C_i\land C_j$ is true, then $\phi$ is well-defined iff $\phi_i=\phi_j$ and each $\phi_k$ for $k\in I$ is well-defined.
\end{theorem}

% piece values = conditions for all pieces, object = values
\subsection{(De)nesting pieces; logical `and'}
The third of several special operations on piecewise objects involves the nesting and denesting of cases; essentially, we decouple the conditions in each piece from one another in order to group and subsequently simply the piece values.
\subsection{Combining, splitting piece conditions; logical `or'}
\subsection{Functions on piecewise objects}
% f(\phi)=yeah; also, explicitly note that a+\phi = ... , a*\phi = ...
\newpage{}