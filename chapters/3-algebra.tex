\section{Algebra on Piecewise Objects}
\subsection{Piece association}
Operations on piecewise objects are fairly straightforward and behave as any other object in the context you're working in (whether this be with the real numbers, complex numbers, certain algebras, spaces, and so on).

What this section aims to do is not to teach you directly how to perform exactly the operations you want to on each piecewise object, but instead provide a basis for which you can base your ideas: no piece of a piecewise object is independent of another. Equivalently, each piece is dependent on each other piece. There is an intuitive reason for this: you are evaluating an object not conditionally, but in full generality; each piece exists, in some sense, `simultaneously'.

It is here we might recognise that this idea could fairly easily lead into combinatorics; an area of maths which deals heavily with the enumeration and construction of such objects.
\subsection{Equality property}
\begin{theorem}
    Consider the following piecewise object:

    $$
        \phi=\pwobj{\varphi_i}{C_i}{i\in I}
    $$

    Then if for all $i,j\in I$ we have that $\varphi_i=\varphi_j$ (that is, all piece values are equal for when $\phi$ is defined), then $\phi=\varphi_i$ for any $i\in I$.

    For example, consider the function $f:(-\infty,0]\to\mathbb{R}$:

    $$
        f(x) = \begin{piecewise}
            0 & x \geq 0 \\
            0 & x < 0
        \end{piecewise}
    $$

    We have that $f(x)=0$. This is because everywhere in $\dom{f}$, or both when $x\geq 0$ and $x < 0$, we have that $f(x)=0$.
\end{theorem}

\subsection{Piece and condition equivalence}
Arguably one of the most fundamental property of piecewise objects is the relationship between each piece's conditions and its respective value. That is, the values for which a piecewise object takes is dependent on their respective conditions, and so can be treated explicitly as such.

\begin{theorem}
    Consider the following piecewise function:

    $$
        f(x) = \pwobj{f_i(x)}{C_i}{i\in I}
    $$

    For all $i\in I$, let us define a substitution $x\sim y_i$, where $\sim$ represents equality under the condition $C_i$. Then we might consider writing $f(x)$ as:

    $$
        f(x) = \pwobj{f_i(y_i)}{C_i}{i\in I}
    $$

    And $f(x)$ may still remain a non-constant function, but in this way we've given it another representation. For some $C_i$ we might just have $x\sim x$ (and is effectively no substitution at all). This substitution can also go the other way, i.e. $y_i\sim x$.
\end{theorem}

\begin{theorem}
    Consider the following piecewise object:

    $$
        \phi=\pwobj{\varphi_i}{C_i}{i\in I}
    $$

    For all $i\in I$, suppose that $C_i \leftrightarrow D_i$ in the context of $\phi$ (i.e. if a function, within its domain, etc.). Then we can rewrite this object as:

    $$
        \phi=\pwobj{\varphi_i}{D_i}{i\in I}
    $$

    That is, we've substituted the set of our conditions for another set of conditions (and again, we may not have substituted all of them). For example, we know that $x=5$ when $(x-5)^2\leq 0$, and so these two conditions are interchangeable.
\end{theorem}

This idea becomes far more important when working using other functions, such as $\max$, $\min$ and $\abs{x}$ (which will be covered later), although it can still be used in other contexts. In essence, for certain classes of piecewise functions, we can build off existing functions to construct not only transformations, but completely new functions and representations.
\begin{example}
    \label{example:equality_1}
    As an example, let us consider the function $f:(-6, 6)\to\mathbb{R}$:

    $$
        f(x) = \begin{piecewise}
            x & x > 5 \\
            5 & x = 5 \\
            x & x < 5
        \end{piecewise}
    $$

    Consider the piece for which $x=5$: note that the piece value is equal to $5$. Since $x=5$ we can replace $5$ with $x$ to get:

    $$
        f(x) = \begin{piecewise}
            x & x > 5 \\
            x & x = 5 \\
            x & x < 5
        \end{piecewise}
    $$

    And by the equality property, we know that this function is equivalent to $f(x)=x$.
\end{example}

Finally, depending on the context of the problem, function or object, we can rewrite conditions in some way to make it more straightforward for us to understand. This becomes relevant in the context of function domains.

\begin{example}
    Let us consider the function from Example $\ref{example:equality_1}$.

    The following two representations are equivalent:

    $$
        f(x) = \begin{piecewise}
            x & x > 5 \\
            5 & x = 5 \\
            x & x < 5
        \end{piecewise}
    $$

    $$
        f(x) = \begin{piecewise}
            x & x\in(5,6) \\
            5 & x = 5 \\
            x & x\in(-6,5)
        \end{piecewise}
    $$

    The reason for this is that for all $x\in(-6,5)$ we have that $x<5$, despite the contrary not being always true. The reason the contrary needn't always be true for all $\mathbb{R}$ is because our function $f$ is defined on the interval $(-6,6)$. This same argument can also be applied to $x\in(5,6)\implies x>5$.
\end{example}

A strong note on functions which are not well-defined: Piecewise objects can easily define not well-defined functions. In these cases, care must be taken when working with pieces in general, values or conditions.

\begin{theorem}
    A piecewise function is only well-defined if all of its piece values denote well-defined functions, and in places where two conditions in separate pieces are true, their respective values must be equal.

    Formally, given the piecewise object $\phi=\pwobj{\varphi_i}{C_i}{i\in I}$ if there exists $i\neq j\in I$ such that $C_i\land C_j$ is true, then $\phi$ is well-defined iff $\varphi_i=\varphi_j$ and each $\varphi_k$ for $k\in I$ is well-defined.

    For example, the following is not well-defined:

    $$
        f(x) = \begin{piecewise}
            1 & x\in\mathbb{Q} \\
            0 & x\in\mathbb{R}\setminus\mathbb{Z}
        \end{piecewise}
    $$

    This is because the latter piece is $0$ for all real numbers that aren't integers, and the first piece is $1$ for all rational numbers; these include non-integers. There are a few ways we could change this function to be well-defined:

    $$
        f(x) = \begin{piecewise}
            1 & x\in\mathbb{Q} \\
            0 & x\in\mathbb{R}\setminus\mathbb{Q}
        \end{piecewise}
    $$

    This is known as the Dirichlet function; it's the indicator function of the rationals (which we'll cover later on). Alternatively:

    $$
        f(x) = \begin{piecewise}
            0 & x\in\mathbb{Q} \\
            0 & x\in\mathbb{R}\setminus\mathbb{Z}
        \end{piecewise}
    $$

    Which would simply make the function $0$ for all $\mathbb{R}$.
\end{theorem}

% piece values = conditions for all pieces, object = values
\subsection{(De)nesting pieces; logical `and'}
The third of several special operations on piecewise objects involves the nesting and denesting of pieces; essentially, we decouple the conditions in each piece from one another in order to group and subsequently simplify the piece values.

\begin{example}
    Consider the following piecewise object:

    $$
        \phi = \begin{piecewise}
            \varphi_1 & C_1 \\
            \varphi_2 & C_2 \\
            \vdots & \vdots \\
            \varphi_n & C_n
        \end{piecewise}
    $$

    Suppose that we have

    $$
        \varphi_1 = \begin{piecewise}
            \mu_1 & D_1 \\
            \mu_2 & D_2
        \end{piecewise}
    $$

    Then we can rewrite $\phi$ as:

    $$
        \phi = \begin{piecewise}
            \begin{piecewise}
                \mu_1 & D_1 \\
                \mu_2 & D_2
            \end{piecewise} & C_1 \\
            \varphi_2 & C_2 \\
            \vdots & \vdots \\
            \varphi_n & C_n
        \end{piecewise}
    $$

    Suppose that $D_1\land C_1$ is true: then it stands to reason that $\phi=\mu_1$. Likewise, if $D_2\land C_1$ then $\phi=\mu_2$. The behaviour is as normal for each other piece. This also means $\phi$ can be represented as:

    $$
        \phi = \begin{piecewise}
            \mu_1 & C_1 & D_1 \\
            \mu_2 & C_1 & D_2 \\
            \varphi_2 & C_2 \\
            \vdots & \vdots \\
            \varphi_n & C_n
        \end{piecewise}
    $$

    And so we might go back and forth between these two forms to represent the same object, keeping in mind each column of the piecewise object, other than the first, represents the conditions under which that piece value is taken (the logical `and').
\end{example}

In general, if we have common, and multiple, conditions for pieces, we should be able to nest, or denest, piecewise objects. This often helps simplify piecewise problems in multiple variables, or single variables with intervals, etc.

\begin{theorem}
    Given the following piecewise object:

    $$
        \phi=\pwobj{\varphi_i}{C_i}{i\in I}
    $$

    If we have that $C_i\leftrightarrow A_i\land B_i$ for some conditions $A_i$, $B_i$, then

    $$
        \phi=\pwobj{\varphi_i}{A_i\land B_i}{i\in I}
    $$

    This is a semi-obvious fact, but becomes useful when coupled with the grouping/nesting of piecewise objects via their piece's conditions. This was also touched on in Chapter \ref{section:notation}.
\end{theorem}
\subsection{Combining, splitting piece conditions; logical `or'}
Finally, we'll look at explicit logical `or' in piece conditions at a basic level and in full generality. This is just as important as the previous section, particularly for grouping.

\begin{theorem}
    Given the following piecewise object (e.g. $A_i\lor B_i\leftrightarrow C_i$):

    $$
        \phi=\pwobj{\varphi_i}{A_i\lor B_i}{i\in I}
    $$

    For some conditions $A_i$, $B_i$, we can rewrite this as the following:

    $$
        \phi=\pwobj{\varphi_i}{A_i}{i\in I}\cup\pwobj{\varphi_i}{B_i}{i\in I}
    $$

    And vice versa.

    We are effectively using the logical `or' condition in each piece to split up the pieces such that each piece only has one condition in the piecewise object representation. For example,

    $$
        f(x) = \begin{piecewise}
            5 & x\geq 5 \lor x\leq -5 \\
            0 & -5<x<5
        \end{piecewise}
    $$

    Is equivalent to:

    $$
        f(x) = \begin{piecewise}
            5 & x\geq 5 \\
            5 & x\leq -5 \\
            0 & -5<x<5
        \end{piecewise}
    $$

    Noting that the last piece cannot be split as it is a logical `and'; that is, $x<5\land x>-5$.
\end{theorem}

\begin{theorem}
    Finally, there is no rule that says pieces are unique in a piecewise object (although generally redundant, duplicate pieces can be useful for grouping once more).

    That is, the following, for some set $J\subseteq I$

    $$
        \phi=\pwobj{\varphi_i}{C_i}{i\in I}
    $$

    is equivalent to:

    $$
        \phi=\pwobj{\varphi_i}{C_i}{i\in I}\cup\pwobj{\varphi_j}{C_j}{j\in J}
    $$

    The reason for this comes down to our well-definition argument as before (equal values).
\end{theorem}
\subsection{Functions on piecewise objects}
Arguably the most important property of piecewise objects is the following:
\begin{theorem}
    Given the following piecewise object:

    $$
        \phi = \pwobj{\varphi_i}{C_i}{i\in I}
    $$

    We have that (when applicable):

    $$
        f(\phi) = \pwobj{f(\varphi_i)}{C_i}{i\in I}
    $$

    A quick intuition is as follows: For all $i\in I$ suppose that at least one $C_i$ is true. Then we have that $\phi=\varphi_i\implies f(\phi)=f(\varphi_i)$. Then, since $C_i\rightarrow f(\phi)=f(\varphi_i)$, we have by definition the above.
\end{theorem}

\begin{theorem}
    Let $\phi=\pwobj{\varphi_i}{C_i}{i\in I}$ and let $a$ be some element in the appropriate domain. We have that:

    \begin{enumerate}
        \item $\phi+a=\pwobj{\varphi_i+a}{C_i}{i\in I}$ (if $\varphi_i+a$ is defined for $i\in I$)
        \item $a\cdot\phi=\pwobj{a\cdot\varphi_i}{C_i}{i\in I}$ (if $a\cdot\varphi_i$ is defined for $i\in I$)
    \end{enumerate}

    And so on, so forth, where applicable.
\end{theorem}

\begin{example}
    Let us define the following function:

    $$
        f(x)=\begin{piecewise}
            3x+2 & x\geq 1 \\
            5 & x\leq 1
        \end{piecewise}
    $$

    We should establish that this function is, in fact, continuous (although this is implied by conditions $x\geq 1$ and $x\leq 1$). For $x\geq 1$ we have that $f(x)=3x+2$ and so $x=1\implies f(1)=5$. Likewise for $x\leq 1$ we have $x=1\implies f(1)=5$.

    We want to rewrite this function as a linear combination (and transformation) of the following function:
    $$
        \abs{x}=\begin{piecewise}
            x & x\geq 0 \\
            -x & x\leq 0
        \end{piecewise}
    $$

    Note that $f(x)$ has conditions $x\geq 1$ and $x\leq 1$, which differs from the conditions in $\abs{x}$. Therefore, we might rewrite these conditions, namely that $x\geq 1\implies x-1\geq 0$ and $x\leq 1\implies x-1\leq 0$. From here, let us define $y=x-1\implies x=y+1$ and substitute into $f(x)$:

    $$
        f(y+1)=\begin{piecewise}
            3(y+1)+2 & y\geq 0\\
            5 & y\leq 0
        \end{piecewise}
    $$

    From here, we should focus on the piece values themselves: Since they differ from the form we want, we should manipulate them until we achieve something that looks like $\abs{y}$. Namely:

    Let us expand the first piece of $f(y+1)$:
    $$
        f(y+1) = \begin{piecewise}
            3y+5 & y\geq 0\\
            5 & y\leq 0
        \end{piecewise}
    $$

    Noting that we have $+5$ as a constant term in each piece, let us `extract' that using our function property:
    $$
        f(y+1) = \begin{piecewise}
            3y & y\geq 0\\
            0 & y\leq 0
        \end{piecewise}+5
    $$

    We're not sure what to do here - we want $y$ in the first piece and $-y$ in the latter, so let us `extract' the coefficient of $y$ again by the same property:
    $$
        f(y+1) = 3\begin{piecewise}
            y & y\geq 0\\
            0 & y\leq 0
        \end{piecewise}+5
    $$
    \addtocounter{example}{-1}
\end{example}
\begin{example}[continued]
    Nearly there. Notice that $y$ is the same as $2y-y$ -- that is, multiply by 2, then subtract a $y$. If we apply this process to the latter piece as well (as is important) we notice that $2\cdot 0-y=-y$, which is what we want. Therefore:
    \begin{align*}
        f(y+1) &= 3\begin{piecewise}
            \frac{1}{2}2y & y\geq 0\\
            0 & y\leq 0
        \end{piecewise}+5\\
        &= \frac{3}{2}\begin{piecewise}
            y+y & y\geq 0\\
            -y+y & y\leq 0
        \end{piecewise}+5 \\
        &= \frac{3}{2}\left(\begin{piecewise}
            y & y\geq 0\\
            -y & y\leq 0
        \end{piecewise}+y\right)+5
    \end{align*}

    Notice now that we have $f(y+1)=\frac{3}{2}\left(\abs{y}+y\right)+5$. Substituting $y=x-1$ back in, we have:

    $$
        f(x)=\frac{3}{2}\left(\abs{x-1}+x-1\right)+5
    $$

    This is not the easiest method of rewriting such piecewise functions, nor is it even remotely scalable. But I would advise you to come up with your own problem with 2 linear functions such as the above (e.g. $4x+10$ for $x\leq -2$ and $2$ for $x\geq -2$) and perform the same process, thinking about why this works, and what sort of linear functions work in this way. Also, consider attempting an example with non-linear functions (and see what the result is). Such examples will help you garner an intuition for manipulating multiple pieces at once.
\end{example}



\begin{example}
    Consider the following truth table:

    \begin{center}
        \begin{tabular}{|c|c|c|}
            \hline
            $x$ & $y$ & $x\land y$ \\
            \hline
            1 & 1 & 1 \\
            1 & 0 & 0 \\
            0 & 1 & 0 \\
            0 & 0 & 0\\
            \hline
        \end{tabular}
    \end{center}

    As it turns out, we can represent $x\land y$ as a piecewise function. That is, $x\land y: \{0,1\}\times\{0,1\}\to\{0,1\}$ such that:
    $$
        x\land y = \begin{piecewise}
            1 & x=1 & y=1 \\
            0 & x=1 & y=0 \\
            0 & x=0 & y=1 \\
            0 & x=0 & y=0
        \end{piecewise}
    $$

    Furthermore, we can perform some fancy algebra magic by nesting pieces (logical AND); we create one piece for when $x=1$ and one for $x=0$:
    $$
        x\land y = \begin{piecewise}
            \begin{piecewise}
                1 & y=1 \\
                0 & y=0
            \end{piecewise} & x = 1 \\
            \begin{piecewise}
                0 & y=1 \\
                0 & y=0
            \end{piecewise} & x = 0
        \end{piecewise}
    $$

    By observation, we might substitute each piece inside the first nested piecewise function with $y$. In our latter nested piecewise function, we simply have $0$ for all possible values of $y$. Therefore:
    $$
        x\land y = \begin{piecewise}
            \begin{piecewise}
                y & y=1 \\
                y & y=0
            \end{piecewise} & x = 1 \\
            0 & x = 0
        \end{piecewise}
    $$

    And performing the same argument as before on our final nested piecewise function, we have:
    $$
        x\land y = \begin{piecewise}
            y & x = 1 \\
            0 & x = 0
        \end{piecewise}
    $$

    Luckily for us, this is fairly straightforward to simplify: `extract' $y$, and then substitute the values with $x$. This gives:
    $$
        x\land y = y\cdot\begin{piecewise}
            x & x = 1 \\
            x & x = 0
        \end{piecewise}
    $$

    And since we have piece values $x$ for all possible values of $x$, we get:
    $$
        x\land y = xy
    $$


\end{example}
\newpage