\section{Common Piecewise Functions}
% abs(x), max(a,b), min(a,b), floor(x), ceil(x), indicator functions
% at some point, cover combinations of piecewise functions etc. AND/OR/etc

% for algebra, use example f(x)=sin(x) for x>=0 and -x for x<=0 using the substitution method. (max/min examples)
So far we've really only used, rather than having properly introduced, the function $\abs{x}$. In this section we'll define a set of piecewise functions that crop up fairly frequently, in many areas of maths, and how we might apply them to the algebra-related problems we've encountered already.

\subsection{Absolute value function}
The absolute value function $\abs{x}$ is a real function defined as follows:

$$
    \abs{x}=\begin{piecewise}
        x & x\geq 0 \\
        -x & x\leq 0
    \end{piecewise}
$$

Alternatively, you might see $\abs{x}$ notated using $\sqrt{x^2}$ for real $x$ and principal square root. This is something which will be introduced and explained in the next section, although $\sqrt{x^2}$ is only one `representation' of $\abs{x}$.

\subsection{The maximum function}
The $\max$ of two variables is a function defined as:

$$
    \max\br{a,b}=\begin{piecewise}
        a & a\geq b \\
        b & a\leq b
    \end{piecewise}
$$

That is, it is a function whose purpose is two return the larger of two numbers. It can also be written as $\max\br{a,b}=\frac{1}{2}\left(a+b+\abs{a-b}\right)$.

\begin{proof}
    We can derive this representation of $\max\br{a,b}$ in much the same way as Example \ref{example:algebra_1}:
    \begin{align*}
        \max\br{a,b} &=\begin{piecewise}
            a & a\geq b\\
            b & a\leq b
        \end{piecewise} \\
        &=\begin{piecewise}
            a-b & a-b\geq 0\\
            0 & a-b\leq 0
        \end{piecewise}+b \\
        &=\frac{1}{2}\left(\begin{piecewise}
            2(a-b) & a-b\geq 0\\
            0 & a-b\leq 0
        \end{piecewise}\right)+b \\
        &=\frac{1}{2}\left(a-b+\begin{piecewise}
            a-b & a-b\geq 0\\
            b-a & a-b\leq 0
        \end{piecewise}\right)+b \\
        &=\frac{1}{2}\left(a-b+\abs{a-b}\right)+b \\
        &=\frac{1}{2}\left(a+b+\abs{a-b}\right) \\
    \end{align*}
\end{proof}

Topping this off, we have that $\abs{x}=\max\br{x,-x}$.

We might also define an extension of the $\max$ function that takes in a finite set; let $\max\br{a}=a$. Then, for a set $S$, we have that:

$$
    \max(S)=\pwobj{x}{x\geq\max(S\setminus\br{x})}{x\in S}
$$

This helps us to derive formulas for $\max\br{a,b,c}$ and so on. Really, in full generality (and slightly out of scope), the max function can be defined as $\sup(S)$ if and only if $\sup(S)\in S$ (otherwise it does not exist at all).

\begin{theorem}
    The union property of the max function is as follows:

    $$
        \max(U\cup V)=\max\br{\max(U),\max(V)}
    $$

    \begin{proof}
        \label{proof:max}
        By definition, the right hand side can be written as:

        $$
            \max\br{\max(U),\max(V)}=\begin{piecewise}
                \max(U) & \max(U)\geq \max(V) \\
                \max(V) & \max(V)\geq \max(U)
            \end{piecewise}
        $$

        We let $x=\max(U\cup V)$. Therefore, $x\geq\max(U)$ and $x\geq\max(V)$, and so we have two cases:

        \begin{enumerate}
            \item $x\in U\implies x=\max(U)$. Since this means $x=\max(U)\geq\max(V)$, we have that $\max\br{\max(U),\max(V)}=x$.
            \item $x\in V\implies x=\max(V)$. Since this means $x=\max(V)\geq\max(U)$, we have that $\max\br{\max(U),\max(V)}=x$.
        \end{enumerate}

        We therefore have that $x=\max\br{\max(U),\max(V)}$.
    \end{proof}

    This property means we can explicitly write the following, for example:
    $$
        \max\br{a,b,c}=\max\br{\max\br{a,b},c}
    $$
    (and hence write it in terms of the absolute value function)
\end{theorem}

\begin{theorem}
    It is worth noting that $f(\max(S))\neq \max(f(S))$, where $f(x)$ is a function and $f(S)$ is $f(x)$ applied elementwise over the set $S$. However:

    \begin{enumerate}
        \item $\max(S)+a=\max(S+a)$ where $S+a$ is elementwise addition by $a$ over $S$.
        \item $c\cdot\max(S)=\max(c\cdot S)$ where $c\geq 0$ and $c\cdot S$ is elementwise multiplication by $c$ over $S$.
        \item $c\cdot\max(S)=\min(c\cdot S)$ where $c\leq 0$ and $c\cdot S$ is elementwise multiplication by $c$ over $S$.
    \end{enumerate}

    Each of these properties can be proven inductively, and such a proof is left to the reader.
\end{theorem}

\subsection{The minimum function}
The $\min$ of two variables is a function defined as:

$$
    \min\br{a,b}=\begin{piecewise}
        b & a\geq b \\
        a & a\leq b
    \end{piecewise}
$$

That is, it is a function whose purpose is two return the smaller of two numbers. It can also be written as $\max\br{a,b}=\frac{1}{2}\left(a+b-\abs{a-b}\right)$. Such a proof of this is identical to the one given in Proof \ref{proof:max}, and you would be encouraged, as the reader, to attempt it explicitly as a personal exercise.

We might also define an (identically motivated) extension of the $\min$ function that takes in a finite set; let $\min\br{a}=a$. Then, for a set $S$, we have that:

$$
    \min(S)=\pwobj{x}{x\leq\min(S\setminus\br{x})}{x\in S}
$$

This helps us to derive formulas for $\min\br{a,b,c}$ and so on. Really, in full generality (and slightly out of scope), the min function can be defined as $\inf(S)$ if and only if $\inf(S)\in S$ (otherwise it does not exist at all).

\begin{theorem}
    The union property of the min function is as follows:

    $$
        \min(U\cup V)=\min\br{\min(U),\min(V)}
    $$

    \begin{proof}
        \label{proof:min}
        By definition, the right hand side can be written as:

        $$
            \min\br{\max(U),\max(V)}=\begin{piecewise}
                \min(V) & \min(U)\geq \min(V) \\
                \min(U) & \min(V)\geq \min(U)
            \end{piecewise}
        $$

        We let $x=\min(U\cup V)$. Therefore, $x\leq\min(U)$ and $x\leq\min(V)$, and so we have two cases:

        \begin{enumerate}
            \item $x\in U\implies x=\min(U)$. Since this means $x=\min(U)\leq\min(V)$, we have that $\min\br{\min(U),\min(V)}=x$.
            \item $x\in V\implies x=\min(V)$. Since this means $x=\min(V)\leq\min(U)$, we have that $\max\br{\min(U),\min(V)}=x$.
        \end{enumerate}

        We therefore have that $x=\min\br{\min(U),\min(V)}$.
    \end{proof}

    This property means we can explicitly write the following, for example:
    $$
        \min\br{a,b,c}=\min\br{\min\br{a,b},c}
    $$
    (and hence write it in terms of the absolute value function)
\end{theorem}

\begin{theorem}
    Just as with the $\max$ function, it is worth noting that $f(\min(S))\neq \min(f(S))$, where $f(x)$ is a function and $f(S)$ is $f(x)$ applied elementwise over the set $S$. However:

    \begin{enumerate}
        \item $\min(S)+a=\min(S+a)$ where $S+a$ is elementwise addition by $a$ over $S$.
        \item $c\cdot\min(S)=\min(c\cdot S)$ where $c\geq 0$ and $c\cdot S$ is elementwise multiplication by $c$ over $S$.
        \item $c\cdot\min(S)=\max(c\cdot S)$ where $c\leq 0$ and $c\cdot S$ is elementwise multiplication by $c$ over $S$.
    \end{enumerate}

    Each of these properties can be proven inductively, and such a proof is left to the reader.
\end{theorem}
\newpage