\section{Common Piecewise Functions}
% abs(x), max(a,b), min(a,b), floor(x), ceil(x), indicator functions
% at some point, cover combinations of piecewise functions etc. AND/OR/etc

% for algebra, use example f(x)=sin(x) for x>=0 and -x for x<=0 using the substitution method. (max/min examples)
So far we've really only used, rather than having properly introduced, the function $\abs{x}$. In this section we'll define a set of piecewise functions that crop up fairly frequently, in many areas of maths, and how we might apply them to the algebra-related problems we've encountered already.

\subsection{Absolute value function}
The absolute value function $\abs{x}$ is a real function defined as follows:

$$
    \abs{x}=\begin{piecewise}
        x & x\geq 0 \\
        -x & x\leq 0
    \end{piecewise}
$$

Alternatively, you might see $\abs{x}$ notated using $\sqrt{x^2}$ for real $x$ and principal square root. This is something which will be introduced and explained in the next section, although $\sqrt{x^2}$ is only one `representation' of $\abs{x}$.

\subsection{The maximum function}
The $\max$ of two variables is a function defined as:

$$
    \max\br{a,b}=\begin{piecewise}
        a & a\geq b \\
        b & a\leq b
    \end{piecewise}
$$

That is, it is a function whose purpose is two return the larger of two numbers. It can also be written as $\max\br{a,b}=\frac{1}{2}\left(a+b+\abs{a-b}\right)$.

\begin{proof}
    We can derive this representation of $\max\br{a,b}$ in much the same way as Example \ref{example:algebra_1}:
    \begin{align*}
        \max\br{a,b} &=\begin{piecewise}
            a & a\geq b\\
            b & a\leq b
        \end{piecewise} \\
        &=\begin{piecewise}
            a-b & a-b\geq 0\\
            0 & a-b\leq 0
        \end{piecewise}+b \\
        &=\frac{1}{2}\left(\begin{piecewise}
            2(a-b) & a-b\geq 0\\
            0 & a-b\leq 0
        \end{piecewise}\right)+b \\
        &=\frac{1}{2}\left(a-b+\begin{piecewise}
            a-b & a-b\geq 0\\
            b-a & a-b\leq 0
        \end{piecewise}\right)+b \\
        &=\frac{1}{2}\left(a-b+\abs{a-b}\right)+b \\
        &=\frac{1}{2}\left(a+b+\abs{a-b}\right) \\
    \end{align*}
\end{proof}

Topping this off, we have that $\abs{x}=\max\br{x,-x}$.

We might also define an extension of the $\max$ function that takes in a finite set; let $\max\br{a}=a$. Then, for a set $S$, we have that:

$$
    \max(S)=\pwobj{x}{x\geq\max(S\setminus\br{x})}{x\in S}
$$

This helps us to derive formulas for $\max\br{a,b,c}$ and so on. Really, in full generality (and slightly out of scope), the max function can be defined as $\sup(S)$ if and only if $\sup(S)\in S$ (otherwise it does not exist at all).

\begin{theorem}
    The union property of the max function is as follows:

    $$
        \max(U\cup V)=\max\br{\max(U),\max(V)}
    $$

    \begin{proof}
        \label{proof:max}
        By definition, the right hand side can be written as:

        $$
            \max\br{\max(U),\max(V)}=\begin{piecewise}
                \max(U) & \max(U)\geq \max(V) \\
                \max(V) & \max(V)\geq \max(U)
            \end{piecewise}
        $$

        We let $x=\max(U\cup V)$. Therefore, $x\geq\max(U)$ and $x\geq\max(V)$, and so we have two cases:

        \begin{enumerate}
            \item $x\in U\implies x=\max(U)$. Since this means $x=\max(U)\geq\max(V)$, we have that $\max\br{\max(U),\max(V)}=x$.
            \item $x\in V\implies x=\max(V)$. Since this means $x=\max(V)\geq\max(U)$, we have that $\max\br{\max(U),\max(V)}=x$.
        \end{enumerate}

        We therefore have that $x=\max\br{\max(U),\max(V)}$.
    \end{proof}

    This property means we can explicitly write the following, for example:
    $$
        \max\br{a,b,c}=\max\br{\max\br{a,b},c}
    $$
    (and hence write it in terms of the absolute value function)
\end{theorem}

\begin{theorem}
    It is worth noting that $f(\max(S))\neq \max(f(S))$, where $f(x)$ is a function and $f(S)$ is $f(x)$ applied elementwise over the set $S$. However:

    \begin{enumerate}
        \item $\max(S)+a=\max(S+a)$ where $S+a$ is elementwise addition by $a$ over $S$.
        \item $c\cdot\max(S)=\max(c\cdot S)$ where $c\geq 0$ and $c\cdot S$ is elementwise multiplication by $c$ over $S$.
        \item $c\cdot\max(S)=\min(c\cdot S)$ where $c\leq 0$ and $c\cdot S$ is elementwise multiplication by $c$ over $S$.
    \end{enumerate}

    Each of these properties can be proven inductively, and such a proof is left to the reader.
\end{theorem}

\subsection{The minimum function}
The $\min$ of two variables is a function defined as:

$$
    \min\br{a,b}=\begin{piecewise}
        b & a\geq b \\
        a & a\leq b
    \end{piecewise}
$$

That is, it is a function whose purpose is two return the smaller of two numbers. It can also be written as $\max\br{a,b}=\frac{1}{2}\left(a+b-\abs{a-b}\right)$. Such a proof of this is identical to the one given in Proof \ref{proof:max}, and you would be encouraged, as the reader, to attempt it explicitly as a personal exercise.

We might also define an (identically motivated) extension of the $\min$ function that takes in a finite set; let $\min\br{a}=a$. Then, for a set $S$, we have that:

$$
    \min(S)=\pwobj{x}{x\leq\min(S\setminus\br{x})}{x\in S}
$$

This helps us to derive formulas for $\min\br{a,b,c}$ and so on. Really, in full generality (and slightly out of scope), the min function can be defined as $\inf(S)$ if and only if $\inf(S)\in S$ (otherwise it does not exist at all).

\begin{theorem}
    The union property of the min function is as follows:

    $$
        \min(U\cup V)=\min\br{\min(U),\min(V)}
    $$

    \begin{proof}
        \label{proof:min}
        By definition, the right hand side can be written as:

        $$
            \min\br{\max(U),\max(V)}=\begin{piecewise}
                \min(V) & \min(U)\geq \min(V) \\
                \min(U) & \min(V)\geq \min(U)
            \end{piecewise}
        $$

        We let $x=\min(U\cup V)$. Therefore, $x\leq\min(U)$ and $x\leq\min(V)$, and so we have two cases:

        \begin{enumerate}
            \item $x\in U\implies x=\min(U)$. Since this means $x=\min(U)\leq\min(V)$, we have that $\min\br{\min(U),\min(V)}=x$.
            \item $x\in V\implies x=\min(V)$. Since this means $x=\min(V)\leq\min(U)$, we have that $\max\br{\min(U),\min(V)}=x$.
        \end{enumerate}

        We therefore have that $x=\min\br{\min(U),\min(V)}$.
    \end{proof}

    This property means we can explicitly write the following, for example:
    $$
        \min\br{a,b,c}=\min\br{\min\br{a,b},c}
    $$
    (and hence write it in terms of the absolute value function)
\end{theorem}

\begin{theorem}
    Just as with the $\max$ function, it is worth noting that $f(\min(S))\neq \min(f(S))$, where $f(x)$ is a function and $f(S)$ is $f(x)$ applied elementwise over the set $S$. However:

    \begin{enumerate}
        \item $\min(S)+a=\min(S+a)$ where $S+a$ is elementwise addition by $a$ over $S$.
        \item $c\cdot\min(S)=\min(c\cdot S)$ where $c\geq 0$ and $c\cdot S$ is elementwise multiplication by $c$ over $S$.
        \item $c\cdot\min(S)=\max(c\cdot S)$ where $c\leq 0$ and $c\cdot S$ is elementwise multiplication by $c$ over $S$.
    \end{enumerate}

    Each of these properties can be proven inductively, and such a proof is left to the reader.
\end{theorem}

\newpage

\subsection{Mixed maximum and minimum}
It is here we note some identities of $\max$ and $\min$ but also extend explicitly our ability to manipulate piecewise functions which would otherwise elude us, per Example \ref{example:algebra_1}.

\begin{example}
    We wish to represent the following function in terms of functions such as $\max$ and $\min$:
    $$
        f(x) = \begin{piecewise}
            \sin(x) & x\geq\pi \\
            \pi - x & x\leq\pi
        \end{piecewise}
    $$

    Notice that $x\geq\pi\iff\max\br{x,\pi}=x$ and $x\leq\pi\iff\min\br{x,\pi}=x$, and so we can provide substitutions for our conditions as so:
    $$
        f(x) = \begin{piecewise}
            \sin(x) & \max\br{x,\pi}=x \\
            \pi - x & \min\br{x,\pi}=x
        \end{piecewise}
    $$

    Furthermore, we have now have substitutions for $x$ in each piece; in each piece value, we perform this substitution for $x$:
    $$
        f(x) = \begin{piecewise}
            \sin(\max\br{x,\pi}) & \max\br{x,\pi}=x \\
            \pi - \min\br{x,\pi} & \min\br{x,\pi}=x
        \end{piecewise}
    $$

    You may be wondering what the point of this is --- well, now that we've performed these substitutions that are only true under each piece, we can `subtract' each piece out (that is, add and substract using our function property):
    \begin{align*}
        f(x) &= \begin{piecewise}
            0-(\pi - \min\br{x,\pi}) & \max\br{x,\pi}=x \\
            0-\sin(\max\br{x,\pi})& \min\br{x,\pi}=x
        \end{piecewise} \\
        &+\sin(\max\br{x,\pi})+\pi - \min\br{x,\pi}
    \end{align*}

    Now, we might evaluate each piece; notice that $\min\br{x,\pi}=\pi$ for $x\geq\pi$ and also that $\max\br{x,\pi}=\pi$ for $x\leq\pi$. Therefore we have:
    \begin{align*}
        f(x) &= \begin{piecewise}
            0 & \max\br{x,\pi}=x \\
            0 & \min\br{x,\pi}=x
        \end{piecewise} \\
        &+\sin(\max\br{x,\pi})+\pi - \min\br{x,\pi}
    \end{align*}

    Finally, simplifying, we're left with:
    $$
        f(x) = \sin(\max\br{x,\pi})-\min\br{x,\pi}+\pi
    $$
\end{example}

\begin{theorem}
    We have the following basic identities to work with using $\max$ and $\min$:
    \begin{enumerate}
        \item $\max\br{a,b}+\min\br{a,b}=a+b$; this result can be proven by definition, or using the $\max$ and $\min$ formulations in terms of the absolute value function.
        \item $\max\br{a,b}-\min\br{a,b}=\abs{a-b}$; this result can be similarly proven as above.
        \item $\abs{x}=\max\br{x,-x}=-\min\br{x,-x}$.
    \end{enumerate}
\end{theorem}

\subsubsection{Clamping function}
The clamping function is a function which restricts a number between an upper and lower bound, as per its definition:

$$
\ell_{a}^{b}(x)=\begin{piecewise}
    b & x\geq b \\
    x & a\leq x\leq b \\
    a & x\leq a
\end{piecewise}
$$

We use this symbol to represent the clamping function as later on it will be given more usage, so it will be useful to have a quick and easy tool (also, this is LaTeX; my hboxes aren't infinite).

\begin{theorem}
    We first provide some properties of the clamping function:
    \begin{enumerate}
        \item $\ell_{a}^{\infty}(x)=\displaystyle\lim_{t\to\infty}\ell_{a}^{t}(x)=\min(x,a)$
        \item $\ell_{-\infty}^{b}(x)=\displaystyle\lim_{t\to\infty}\ell_{-t}^{b}=\max(x,b)$
        \item $-\ell_{a}^{b}(x)=\ell_{-b}^{-a}(-x)$
        \item $c\cdot\ell_{a}^{b}(x)=\ell_{ac}^{bc}(cx)$, for $c\geq 0$
        \item $\ell_{a}^{b}(x)+k=\ell_{a+k}^{b+k}(x+k)$
    \end{enumerate}
\end{theorem}
\begin{theorem}
    The clamping function can be written as any of the following:
    \begin{enumerate}
        \item   $\ell_{a}^{b}(x)=\min\br{\max\br{x,a},b}$
        \item   $\ell_{a}^{b}(x)=\max\br{\min\br{x,b},a}$
        \item   $\ell_{a}^{b}(x)=\frac{1}{2}\left(a+b+\abs{x-a}-\abs{x-b}\right)$
    \end{enumerate}

    We shall give proofs of the first and last of these formulations:
    \begin{proof}
        We begin by using the definition of $\ell_{a}^{b}(x)$:
        $$
            \ell_{a}^{b}(x)=\begin{piecewise}
                b & x\geq b \\
                x & a\leq x\leq b \\
                a & x\leq a
            \end{piecewise}
        $$

        We nest a piece under the conditions $x\geq a$ and $x\leq a$ in order to simplify the conditions we're working with:
        \begin{align*}
            \ell_{a}^{b}(x) &=\begin{piecewise}
                \begin{piecewise}
                    b & x\geq b \\
                    x & x\leq b
                \end{piecewise} & x\geq a \\
                a & x\leq a
            \end{piecewise} \\
            &= \begin{piecewise}
                \min\br{x,b} & x\geq a \\
                a & x\leq a
            \end{piecewise}
        \end{align*}

        Now using the definition of $\max\br{x,a}$ we substitute $x\geq a$ with $\max\br{x,a}=x$ and likewise with $x\leq a$, to give:
        $$
            \ell_{a}^{b}(x)=\begin{piecewise}
                \min\br{x,b} & \max\br{x,a}=x \\
                a & \max\br{x,a}=a
            \end{piecewise}
        $$

        We then use the first piece's condition to substitute the value of $x$ with $\max\br{x,a}$,
        $$
            \ell_{a}^{b}(x)=\begin{piecewise}
                \min\br{\max\br{x,a},b} & \max\br{x,a}=x \\
                a & \max\br{x,a}=a
            \end{piecewise}
        $$

        Substracting out $\min\br{\max\br{x,a},b}$ gives us
        $$
            \ell_{a}^{b}(x)=\begin{piecewise}
                0 & x\geq a \\
                a-\min\br{\max\br{x,a},b} & x\leq a
            \end{piecewise}+\min\br{\max\br{x,a},b}
        $$

        And evaluating the second piece (since $x\leq a\leq b$) gives us $0$, leaving us with
        $$
            \ell_{a}^{b}(x)=\min\br{\max\br{x,a},b}
        $$
    \end{proof}
    \addtocounter{theorem}{-1}
\end{theorem}

\begin{theorem}[continued]
    \begin{proof}
        We now set out to prove the last representation of $\ell_{a}^{b}(x)$.

        Using our previous derivation, we apply the addition property of $\min$ to give us:
        $$
            \min\br{\max\br{x,a},b}=\min\br{0,b-\max\br{x,a}}+\max\br{x,a}
        $$

        Which simplifies to:
        \begin{align*}
            \ell_{a}^{b}(x) &= \min\br{0,\min\br{b-x,b-a}}+\max\br{x,a} \\
                            &=\min\br{0,b-x,b-a}+\max\br{x,a}
        \end{align*}

        Since $b-a\geq 0$, we're left with:
        $$
            \ell_{a}^{b}(x)=\min\br{b-x,0}+\max\br{x,a}
        $$

        Using the absolute-value representations of $\max$ and $\min$ and simplifying, we get that
        $$
            \ell_{a}^{b}(x)=\frac{1}{2}\left(a+b+\abs{x-a}-\abs{x-b}\right)
        $$
    \end{proof}
\end{theorem}


\newpage