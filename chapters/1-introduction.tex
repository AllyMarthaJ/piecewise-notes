\section{Introduction}
These notes serve as a single source of knowledge in which those reading should ideally be able to garner a deeper understanding of the thoughts and ideas I have had and continue to have.

To consider something piecewise is to enumerate something under differing conditions. For example, one might consider the function $\abs{x}$ piecewise, namely when $x\geq 0$ or $x\leq 0$.

An object which is piecewise is an object which contains a set of pieces, which themselves contain values and conditions under which those values are taken for the overall object. Continuing with the example of $\abs{x}$, we have the piece value $x$ for when $x\geq 0$ (which forms a piece), and $-x$ for when $x\leq 0$.

We shall develop ideas such as above more explicitly throughout these notes. These ideas often intersect with other areas of maths, which one might be familiar with - but if not, don't panic; such ideas are not strictly foundational to the presented notes. 

Furthermore, these notes focus on construction (not to be confused with constructivism); rather than evaluating existing concepts or formulas, we focus on deriving existing or new tools, ideas and formulas. For you, reader, this must be a process should become familiar with, and rather than just reading these notes, attempt to follow along by hand, and construct your own objects using the ideas presented here. We also stress that with ideas that intersect with more mainstream mathematics, that existing concepts be used to evaluate the validity of the constructions.