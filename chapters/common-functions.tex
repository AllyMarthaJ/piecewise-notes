\section{Common Piecewise Functions}
% abs(x), max(a,b), min(a,b), floor(x), ceil(x), indicator functions
% at some point, cover combinations of piecewise functions etc. AND/OR/etc

% for algebra, use example f(x)=sin(x) for x>=0 and -x for x<=0 using the substitution method. (max/min examples)
So far we've really only used, rather than having properly introduced, the function $\abs{x}$. In this section we'll define a set of piecewise functions that crop up fairly frequently, in many areas of maths, and how we might apply them to the algebra-related problems we've encountered already.

\subsection{Absolute value function}
The absolute value function $\abs{x}$ is a real function defined as follows:

$$
    \abs{x}=\begin{piecewise}
        x & x\geq 0 \\
        -x & x\leq 0
    \end{piecewise}
$$

Alternatively, you might see $\abs{x}$ notated using $\sqrt{x^2}$ for real $x$ and principal square root. This is something which will be introduced and explained in the next section, although $\sqrt{x^2}$ is only one `representation' of $\abs{x}$.

\subsection{The maximum function}
\label{section:max}
The $\max$ of two variables is a function defined as:

$$
    \max\br{a,b}=\begin{piecewise}
        a & a\geq b \\
        b & a\leq b
    \end{piecewise}
$$

That is, it is a function whose purpose is two return the larger of two numbers. It can also be written as $\max\br{a,b}=\frac{1}{2}\left(a+b+\abs{a-b}\right)$.

\begin{proof}
    We can derive this representation of $\max\br{a,b}$ in much the same way as Example \ref{example:algebra_1}:
    \begin{align*}
        \max\br{a,b} &=\begin{piecewise}
            a & a\geq b\\
            b & a\leq b
        \end{piecewise} \\
        &=\begin{piecewise}
            a-b & a-b\geq 0\\
            0 & a-b\leq 0
        \end{piecewise}+b \\
        &=\frac{1}{2}\left(\begin{piecewise}
            2(a-b) & a-b\geq 0\\
            0 & a-b\leq 0
        \end{piecewise}\right)+b \\
        &=\frac{1}{2}\left(a-b+\begin{piecewise}
            a-b & a-b\geq 0\\
            b-a & a-b\leq 0
        \end{piecewise}\right)+b \\
        &=\frac{1}{2}\left(a-b+\abs{a-b}\right)+b \\
        &=\frac{1}{2}\left(a+b+\abs{a-b}\right) \\
    \end{align*}
\end{proof}

Topping this off, we have that $\abs{x}=\max\br{x,-x}$.

We might also define an extension of the $\max$ function that takes in a finite set; let $\max\br{a}=a$. Then, for a set $S$, we have that:

$$
    \max(S)=\pwobj{x}{x\geq\max(S\setminus\br{x})}{x\in S}
$$

This helps us to derive formulas for $\max\br{a,b,c}$ and so on. Really, in full generality (and slightly out of scope), the max function can be defined as $\sup(S)$ if and only if $\sup(S)\in S$ (otherwise it does not exist at all).

\begin{theorem}
    The union property of the max function is as follows:

    $$
        \max(U\cup V)=\max\br{\max(U),\max(V)}
    $$

    \begin{proof}
        \label{proof:max}
        By definition, the right hand side can be written as:

        $$
            \max\br{\max(U),\max(V)}=\begin{piecewise}
                \max(U) & \max(U)\geq \max(V) \\
                \max(V) & \max(V)\geq \max(U)
            \end{piecewise}
        $$

        We let $x=\max(U\cup V)$. Therefore, $x\geq\max(U)$ and $x\geq\max(V)$, and so we have two cases:

        \begin{enumerate}
            \item $x\in U\implies x=\max(U)$. Since this means $x=\max(U)\geq\max(V)$, we have that $\max\br{\max(U),\max(V)}=x$.
            \item $x\in V\implies x=\max(V)$. Since this means $x=\max(V)\geq\max(U)$, we have that $\max\br{\max(U),\max(V)}=x$.
        \end{enumerate}

        We therefore have that $x=\max\br{\max(U),\max(V)}$.
    \end{proof}

    This property means we can explicitly write the following, for example:
    $$
        \max\br{a,b,c}=\max\br{\max\br{a,b},c}
    $$
    (and hence write it in terms of the absolute value function)
\end{theorem}

\begin{theorem}
    It is worth noting that $f(\max(S))\neq \max(f(S))$, where $f(x)$ is a function and $f(S)$ is $f(x)$ applied elementwise over the set $S$. However:

    \begin{enumerate}
        \item $\max(S)+a=\max(S+a)$ where $S+a$ is elementwise addition by $a$ over $S$.
        \item $c\cdot\max(S)=\max(c\cdot S)$ where $c\geq 0$ and $c\cdot S$ is elementwise multiplication by $c$ over $S$.
        \item $c\cdot\max(S)=\min(c\cdot S)$ where $c\leq 0$ and $c\cdot S$ is elementwise multiplication by $c$ over $S$.
    \end{enumerate}

    Each of these properties can be proven inductively, and such a proof is left to the reader.
\end{theorem}

\subsection{The minimum function}
\label{section:min}
The $\min$ of two variables is a function defined as:

$$
    \min\br{a,b}=\begin{piecewise}
        b & a\geq b \\
        a & a\leq b
    \end{piecewise}
$$

That is, it is a function whose purpose is two return the smaller of two numbers. It can also be written as $\max\br{a,b}=\frac{1}{2}\left(a+b-\abs{a-b}\right)$. Such a proof of this is identical to the one given in Proof \ref{proof:max}, and you would be encouraged, as the reader, to attempt it explicitly as a personal exercise.

We might also define an (identically motivated) extension of the $\min$ function that takes in a finite set; let $\min\br{a}=a$. Then, for a set $S$, we have that:

$$
    \min(S)=\pwobj{x}{x\leq\min(S\setminus\br{x})}{x\in S}
$$

This helps us to derive formulas for $\min\br{a,b,c}$ and so on. Really, in full generality (and slightly out of scope), the min function can be defined as $\inf(S)$ if and only if $\inf(S)\in S$ (otherwise it does not exist at all).

\begin{theorem}
    The union property of the min function is as follows:

    $$
        \min(U\cup V)=\min\br{\min(U),\min(V)}
    $$

    \begin{proof}
        \label{proof:min}
        By definition, the right hand side can be written as:

        $$
            \min\br{\max(U),\max(V)}=\begin{piecewise}
                \min(V) & \min(U)\geq \min(V) \\
                \min(U) & \min(V)\geq \min(U)
            \end{piecewise}
        $$

        We let $x=\min(U\cup V)$. Therefore, $x\leq\min(U)$ and $x\leq\min(V)$, and so we have two cases:

        \begin{enumerate}
            \item $x\in U\implies x=\min(U)$. Since this means $x=\min(U)\leq\min(V)$, we have that $\min\br{\min(U),\min(V)}=x$.
            \item $x\in V\implies x=\min(V)$. Since this means $x=\min(V)\leq\min(U)$, we have that $\max\br{\min(U),\min(V)}=x$.
        \end{enumerate}

        We therefore have that $x=\min\br{\min(U),\min(V)}$.
    \end{proof}

    This property means we can explicitly write the following, for example:
    $$
        \min\br{a,b,c}=\min\br{\min\br{a,b},c}
    $$
    (and hence write it in terms of the absolute value function)
\end{theorem}

\begin{theorem}
    Just as with the $\max$ function, it is worth noting that $f(\min(S))\neq \min(f(S))$, where $f(x)$ is a function and $f(S)$ is $f(x)$ applied elementwise over the set $S$. However:

    \begin{enumerate}
        \item $\min(S)+a=\min(S+a)$ where $S+a$ is elementwise addition by $a$ over $S$.
        \item $c\cdot\min(S)=\min(c\cdot S)$ where $c\geq 0$ and $c\cdot S$ is elementwise multiplication by $c$ over $S$.
        \item $c\cdot\min(S)=\max(c\cdot S)$ where $c\leq 0$ and $c\cdot S$ is elementwise multiplication by $c$ over $S$.
    \end{enumerate}

    Each of these properties can be proven inductively, and such a proof is left to the reader.
\end{theorem}

\subsection{Mixed maximum and minimum}
It is here we note some identities of $\max$ and $\min$ but also extend explicitly our ability to manipulate piecewise functions which would otherwise elude us, per Example \ref{example:algebra_1}.

\begin{example}
    \label{example:gluing_basic}
    We wish to represent the following function in terms of functions such as $\max$ and $\min$:
    $$
        f(x) = \begin{piecewise}
            \sin(x) & x\geq\pi \\
            \pi - x & x\leq\pi
        \end{piecewise}
    $$

    Notice that $x\geq\pi\iff\max\br{x,\pi}=x$ and $x\leq\pi\iff\min\br{x,\pi}=x$, and so we can provide substitutions for our conditions as so:
    $$
        f(x) = \begin{piecewise}
            \sin(x) & \max\br{x,\pi}=x \\
            \pi - x & \min\br{x,\pi}=x
        \end{piecewise}
    $$

    Furthermore, we have now have substitutions for $x$ in each piece; in each piece value, we perform this substitution for $x$:
    $$
        f(x) = \begin{piecewise}
            \sin(\max\br{x,\pi}) & \max\br{x,\pi}=x \\
            \pi - \min\br{x,\pi} & \min\br{x,\pi}=x
        \end{piecewise}
    $$

    You may be wondering what the point of this is --- well, now that we've performed these substitutions that are only true under each piece, we can `subtract' each piece out (that is, add and substract using our function property):
    \begin{align*}
        f(x) &= \begin{piecewise}
            0-(\pi - \min\br{x,\pi}) & \max\br{x,\pi}=x \\
            0-\sin(\max\br{x,\pi})& \min\br{x,\pi}=x
        \end{piecewise} \\
        &+\sin(\max\br{x,\pi})+\pi - \min\br{x,\pi}
    \end{align*}

    Now, we might evaluate each piece; notice that $\min\br{x,\pi}=\pi$ for $x\geq\pi$ and also that $\max\br{x,\pi}=\pi$ for $x\leq\pi$. Therefore we have:
    \begin{align*}
        f(x) &= \begin{piecewise}
            0 & \max\br{x,\pi}=x \\
            0 & \min\br{x,\pi}=x
        \end{piecewise} \\
        &+\sin(\max\br{x,\pi})+\pi - \min\br{x,\pi}
    \end{align*}

    Finally, simplifying, we're left with:
    $$
        f(x) = \sin(\max\br{x,\pi})-\min\br{x,\pi}+\pi
    $$
\end{example}

\begin{theorem}
    We have the following basic identities to work with using $\max$ and $\min$:
    \begin{enumerate}
        \item $\max\br{a,b}+\min\br{a,b}=a+b$; this result can be proven by definition, or using the $\max$ and $\min$ formulations in terms of the absolute value function.
        \item $\max\br{a,b}-\min\br{a,b}=\abs{a-b}$; this result can be similarly proven as above.
        \item $\abs{x}=\max\br{x,-x}=-\min\br{x,-x}$.
    \end{enumerate}
\end{theorem}

\subsubsection{Clamping function}
\label{section:clamping_function}
The clamping function is a function which restricts a number between an upper and lower bound, as per its definition:

$$
\ell_{a}^{b}(x)=\begin{piecewise}
    b & x\geq b \\
    x & a\leq x\leq b \\
    a & x\leq a
\end{piecewise}
$$

We use this symbol to represent the clamping function as later on it will be given more usage, so it will be useful to have a quick and easy tool (also, this is LaTeX; my hboxes aren't infinite).

\begin{theorem}
    We first provide some properties of the clamping function:
    \begin{enumerate}
        \item $\ell_{a}^{\infty}(x)=\displaystyle\lim_{t\to\infty}\ell_{a}^{t}(x)=\max\br{x,a}$
        \item $\ell_{-\infty}^{b}(x)=\displaystyle\lim_{t\to\infty}\ell_{-t}^{b}=\min\br{x,b}$
        \item $-\ell_{a}^{b}(x)=\ell_{-b}^{-a}(-x)$
        \item $c\cdot\ell_{a}^{b}(x)=\ell_{ac}^{bc}(cx)$, for $c\geq 0$
        \item $\ell_{a}^{b}(x)+k=\ell_{a+k}^{b+k}(x+k)$
    \end{enumerate}
\end{theorem}
\begin{theorem}
    The clamping function can be written as any of the following:
    \begin{enumerate}
        \item   $\ell_{a}^{b}(x)=\min\br{\max\br{x,a},b}$
        \item   $\ell_{a}^{b}(x)=\max\br{\min\br{x,b},a}$
        \item   $\ell_{a}^{b}(x)=\frac{1}{2}\left(a+b+\abs{x-a}-\abs{x-b}\right)$
    \end{enumerate}

    We shall give proofs of the first and last of these formulations:
    \begin{proof}
        We begin by using the definition of $\ell_{a}^{b}(x)$:
        $$
            \ell_{a}^{b}(x)=\begin{piecewise}
                b & x\geq b \\
                x & a\leq x\leq b \\
                a & x\leq a
            \end{piecewise}
        $$

        We nest a piece under the conditions $x\geq a$ and $x\leq a$ in order to simplify the conditions we're working with:
        \begin{align*}
            \ell_{a}^{b}(x) &=\begin{piecewise}
                \begin{piecewise}
                    b & x\geq b \\
                    x & x\leq b
                \end{piecewise} & x\geq a \\
                a & x\leq a
            \end{piecewise} \\
            &= \begin{piecewise}
                \min\br{x,b} & x\geq a \\
                a & x\leq a
            \end{piecewise}
        \end{align*}

        Now using the definition of $\max\br{x,a}$ we substitute $x\geq a$ with $\max\br{x,a}=x$ and likewise with $x\leq a$, to give:
        $$
            \ell_{a}^{b}(x)=\begin{piecewise}
                \min\br{x,b} & \max\br{x,a}=x \\
                a & \max\br{x,a}=a
            \end{piecewise}
        $$

        We then use the first piece's condition to substitute the value of $x$ with $\max\br{x,a}$,
        $$
            \ell_{a}^{b}(x)=\begin{piecewise}
                \min\br{\max\br{x,a},b} & \max\br{x,a}=x \\
                a & \max\br{x,a}=a
            \end{piecewise}
        $$

        Substracting out $\min\br{\max\br{x,a},b}$ gives us
        $$
            \ell_{a}^{b}(x)=\begin{piecewise}
                0 & x\geq a \\
                a-\min\br{\max\br{x,a},b} & x\leq a
            \end{piecewise}+\min\br{\max\br{x,a},b}
        $$

        And evaluating the second piece (since $x\leq a\leq b$) gives us $0$, leaving us with
        $$
            \ell_{a}^{b}(x)=\min\br{\max\br{x,a},b}
        $$
    \end{proof}

    \begin{proof}
        We now set out to prove the last representation of $\ell_{a}^{b}(x)$.

        Using our previous derivation, we apply the addition property of $\min$ to give us:
        $$
            \min\br{\max\br{x,a},b}=\min\br{0,b-\max\br{x,a}}+\max\br{x,a}
        $$

        Which simplifies to:
        \begin{align*}
            \ell_{a}^{b}(x) &= \min\br{0,b+\min\br{-x,-a}}+\max\br{x,a} \\
                            &= \min\br{0,\min\br{b-x,b-a}}+\max\br{x,a} \\
                            &=\min\br{0,b-x,b-a}+\max\br{x,a}
        \end{align*}

        Since $b-a\geq 0$, we're left with:
        $$
            \ell_{a}^{b}(x)=\min\br{b-x,0}+\max\br{x,a}
        $$

        Using the absolute value representations of $\max$ and $\min$ and simplifying, we get that
        $$
            \ell_{a}^{b}(x)=\frac{1}{2}\left(a+b+\abs{x-a}-\abs{x-b}\right)
        $$
    \end{proof}
\end{theorem}

\begin{example}
    \label{example:algebra_clamping}
    We want to consider the following function, rewriting it in terms of $\max$, $\min$:

    $$
        f(x) = \begin{piecewise}
            x & x>1 \\
            1 & -1\leq x\leq 1 \\
            -x & x<-1
        \end{piecewise}
    $$

    This is not in the form we want in order to manipulate it, so we consider that this function is, in fact, continuous. It stands to reason, therefore, that $x>1\iff x\geq 1$ and $x<-1\iff x\leq -1$. Therefore:

    $$
        f(x) = \begin{piecewise}
            x & x\geq 1 \\
            1 & -1\leq x\leq 1 \\
            -x & x\leq-1
        \end{piecewise}
    $$

    Now, we note that $x\geq 1$ is equivalent to $\ell_{1}^{\infty}(x)=x$, $-1\leq x\leq 1$ is equivalent to $\ell_{-1}^{1}(x)=x$ and $x\leq -1$ is equivalent to $\ell_{-\infty}^{-1}(x)=x$.

    With our equivalences, we can now make the respective substitutions in both the conditions and piece values, as with previous examples (although we needn't really keep the conditions in this form):

    $$
        f(x) = \begin{piecewise}
            \ell_{1}^{\infty}(x) & \ell_{1}^{\infty}(x)=x \\
            1 & \ell_{-1}^{1}(x)=x \\
            -\ell_{-\infty}^{-1}(x) & \ell_{-\infty}^{-1}(x)=x
        \end{piecewise}
    $$

    Now we `extract' our piece values by addition:
    $$
        f(x) = \paren*{\begin{piecewise}
            -(1-\ell_{-\infty}^{-1}(x)) & x\geq 1 \\
            -(\ell_{1}^{\infty}(x)-\ell_{-\infty}^{-1}(x)) & -1\leq x\leq 1 \\
            -(\ell_{1}^{\infty}(x)+1) & x\leq -1
        \end{piecewise}}+\paren*{\ell_{1}^{\infty}(x)+1-\ell_{-\infty}^{-1}(x)}
    $$

    Evaluting each piece value using the definition of the clamping function gives us:
    $$
        f(x) = \paren*{\begin{piecewise}
            -2 & x\geq 1 \\
            -2 & -1\leq x\leq 1 \\
            -2 & x\leq -1
        \end{piecewise}}+\paren*{\ell_{1}^{\infty}(x)+1-\ell_{-\infty}^{-1}(x)}
    $$

    Finally simplifying, we have:
    $$
        f(x) = \ell_{1}^{\infty}(x)-\ell_{-\infty}^{-1}(x)-1
    $$

    Notice that, in this example, we don't have any $\ell_{-1}^{1}(x)$ because, in fact, $1$ is a constant function. Rewriting our clamping functions in terms of $\max$ and $\min$ per the limit identities gives us:
    $$
        f(x) = \max\br{x,1}-\min\br{x,-1}-1
    $$
\end{example}

The clamping function is useful for formulating single expressions for continuous piecewise functions as we've seen above. If the function is not continuous, we may not actually have much luck in using it. In fact, we'll derive the general `Gluing Formula' soon within these notes.

\subsection{The sign function}
The sign function is an interesting function in that it doesn't have a single explicit definition, but instead satisfies the following relation:
$$
    x\cdot\sgn{x}=\abs{x}
$$

This can be true for some domain $D\subseteq\mathbb{R}$, such as $\mathbb{R}\setminus\br{0}$. For the purposes of these notes, however, we shall define $\sgn{x}$ as the following:
$$
    \sgn{x}=\begin{piecewise}
        1 & x>0 \\
        0 & x=0 \\
        -1 & x<0
    \end{piecewise}
$$

This definition does indeed satisfy the definition of $\abs{x}$, noting that we can write $\abs{x}$ in the following explicit form\footnotemark:
$$
    \abs{x}=\begin{piecewise}
        x & x>0 \\
        0 & x=0 \\
        -x & x<0
    \end{piecewise}
$$

\footnotetext{It is a good exercise to reconcile this form of $\abs{x}$ with the previous forms displayed in these notes, using piecewise function properties (and when and why each form may be useful to us).}

It is useful to note that the sign function is discontinuous; we can express discontinuous piecewise functions in terms of the sign function, although this is not common (we instead opt for step functions there, although step functions are just transformations of the sign function).

\begin{theorem}
    \label{theorem:odd_sign}
    Given an odd function $f:A\to\mathbb{R}$ where $A=-A$, we have that $f(x)\sgn{x}=f(\abs{x})$ and $f(\abs{x})\sgn{x}=f(x)$.

    \begin{proof}
        We begin with the left hand side of the first property and perform the usual steps:
        \begin{align*}
            f(x)\sgn{x} &= f(x)\begin{piecewise}
                -1 & x<0 \\
                0 & x=0 \\
                1 & x>0
            \end{piecewise}\\
            &= \begin{piecewise}
                -f(x) & x<0 \\
                0 & x=0 \\
                f(x) & x>0
            \end{piecewise}\\
            &= \begin{piecewise}
                f(-x) & x<0 \\
                0 & x=0 \\
                f(x) & x>0
            \end{piecewise}\\
            &= f\paren*{\begin{piecewise}
                -x & x<0 \\
                0 & x=0 \\
                x & x>0
            \end{piecewise}}\\
            &= f(\abs{x})
        \end{align*}
    \end{proof}

    The third and fourth steps of this proof follows from $f(x)$ being odd; that is, $f(0)=0$ and $f(-x)=-f(x)$.

    \begin{proof}
        The second property is similar to the first:
        \begin{align*}
            f(\abs{x})\sgn{x} &=\begin{piecewise}
                f(-x) & x<0 \\
                f(0) & x=0 \\
                f(x) & x>0
            \end{piecewise}\cdot\begin{piecewise}
                -1 & x<0 \\
                0 & x=0 \\
                1 & x>0
            \end{piecewise} \\
            &= \begin{piecewise}
                -f(-x) & x<0 \\
                0 & x=0 \\
                f(x) & x>0
            \end{piecewise} \\
            &= \begin{piecewise}
                f(x) & x<0 \\
                f(x) & x=0 \\
                f(x) & x>0
            \end{piecewise} \\
            &= f(x)
        \end{align*}
    \end{proof}
\end{theorem}

\subsection{Step functions}
Step functions are useful functions which, usually in calculus at higher levels, allow us to provide a form in which to express a piecewise function regardless of continuity or not. We look at the Heaviside step function, which, like the sign function, doesn't have a single explicit formulation, but instead satisfies:
$$
    x\cdot\step{x}=\frac{1}{2}\paren*{x+\abs{x}}
$$

The right hand side of this relation is known as the ramp function and can be expressed as $\max\br{x,0}$.

For the purposes of these notes we might define the Heaviside step function as the following:
$$
    \step{x}=\begin{piecewise}
        1 & x>0 \\
        0 & x<0
    \end{piecewise}
$$

\begin{theorem}
    We can express $\step{x}$ in terms of $\sgn{x}$;

    $$
        \step{x}=\frac{1}{2}\paren{1+\sgn{x}}
    $$

    This can be shown using the same techniques found in Example \ref{example:algebra_1}.
\end{theorem}

\begin{example}
    We wish to express the following function (similar to that found in \ref{example:algebra_clamping}) in terms of step functions:
    $$
        f(x) = \begin{piecewise}
            x & x>1 \\
            1 & -1<x<1 \\
            -x & x<-1
        \end{piecewise}
    $$

    In order to simplify this down, let us nest piecewise functions on the conditions $x>-1$ and $x<-1$:
    $$
        f(x) = \begin{piecewise}
            \begin{piecewise}
                x & x>1 \\
                1 & x<1
            \end{piecewise} & x>-1 \\
            -x & x<-1
        \end{piecewise}
    $$

    We now wish to write the inner piecewise function in terms of the step function: to do so, we `subtract' our second piece, $1$ out and then factor out $x-1$, as follows:
    $$
        f(x) = \begin{piecewise}
            (x-1)\begin{piecewise}
                1 & x>1 \\
                0 & x<1
            \end{piecewise}+1 & x>-1 \\
            -x & x<-1
        \end{piecewise}
    $$

    Therefore, the inner piecewise function is equivalent to $\step{x-1}$:
    $$
        f(x) = \begin{piecewise}
            (x-1)\step{x-1}+1 & x>-1 \\
            -x & x<-1
        \end{piecewise}
    $$

    Instead of repeating this process with the outer piecewise function, we should instead subtract out the first piece, since we have a step function inside that:
    $$
        f(x) = (x-1)\step{x-1}+1+\begin{piecewise}
            0 & x>-1 \\
            -x-(x-1)\step{x-1}-1 & x<-1
        \end{piecewise}
    $$

    And from there, evaluate the second piece (wherein $\step{x-1}=0$ since $x<-1$):
    $$
        f(x) = (x-1)\step{x-1}+1+\begin{piecewise}
            0 & x>-1 \\
            -x-1 & x<-1
        \end{piecewise}
    $$

    We can factor out $(-x-1)$ again and rewrite $x<-1$ as $-x-1>0$:
    $$
        f(x) = (x-1)\step{x-1}+1+(-x-1)\begin{piecewise}
            0 & -x-1<0 \\
            1 & -x-1>0
        \end{piecewise}
    $$
    \addtocounter{example}{-1}

    We therefore have that:
    $$
        f(x) = (x-1)\step{x-1}+(-x-1)\step{-x-1}+1
    $$

    The astute may have noticed that, in fact, this function can be rewritten directly as a result of the definition of the step function:
    $$
        f(x) = \max\br{x-1,0}+\max\br{-x-1,0}+1
    $$

    We have, as a result, that $f(x)$ is continuous (which we already knew from the previous example), as the $\max$ function is continuous. Note however that unlike the previous example, even though we can write this function in this form, that $f(-1)$ and $f(1)$ are not defined.

    Explicitly, we write $f:\mathbb{R}\setminus\br{-1,1}\to\mathbb{R}$.
\end{example}

This process is generalisable as with our continuous gluing function (i.e. through repeating nesting). I will, however, leave this process up to the reader.

\subsection{The floor and ceiling functions}
We've already introduced the floor function once previously, however we shall do so again here for the sake of completeness:
$$
\floor{x}=\pwobj{n}{x\in[n,n+1)}{n\in\mathbb{Z}}
$$

We can alternatively write the floor function using the $\max$ function of an infinite set:
$$
\floor{x}=\max\br{n\in\mathbb{Z}\mid n\leq x}
$$

That is, the floor function is the greatest integer $n$ such that $n$ is less than or equal to $x$.

The ceiling function is similarly defined piecewise;
$$
\ceil{x}=\pwobj{n}{x\in(n-1,n]}{n\in\mathbb{Z}}
$$

or alternatively using the min function:
$$
\ceil{x}=\min\br{n\in\mathbb{Z}\mid n\geq x}
$$

That is, the ceiling function is the smallest integer $n$ such that $n$ is larger than or equal to $x$.

\subsection{The modulo operation/function}
The modulo operation is an operation used when computing the remainder in division. As such it has several ambiguities regarding the sign of the divisor and remainder. With that being said, for the purposes of these notes we shall define it per the following, for $y>0$:
$$
    x\bmod y = \pwobj{x-ny}{x\in[ny,(n+1)y)}{n\in\mathbb{Z}}
$$
Which gives the positive remainder for positive divisors.

\begin{theorem}
    The modulo operation as defined in these notes can be written as:
    $$
        x\bmod y = x-y\floor*{\frac{x}{y}}
    $$

    \begin{proof}
        We begin by using the definition of $x\bmod y$, and subsequently applying the function property to our piece values to write them in terms of $n$:
        $$
            x\bmod y = x-y\pwobj{n}{x\in[ny,(n+1)y)}{n\in\mathbb{Z}}
        $$

        Notice the piecewise function is close to the definition of the floor function. We rewrite our conditions accordingly:
        $$
            x\bmod y = x-y\pwobj{n}{\frac{x}{y}\in[n,n+1)}{n\in\mathbb{Z}}
        $$

        The piecewise object now matches the definition for $\floor*{\frac{x}{y}}$ and therefore we have:
        $$
            x\bmod y = x-y\floor*{\frac{x}{y}}
        $$

        (If we like, we can also use this definition for modulo a negative number.)
    \end{proof}
\end{theorem}

\subsection{Characteristic functions and Iverson brackets}
Briefly, a characteristic (or indicator) function for a set $S$ is a function defined such that:
$$
    \mathbf{1}_{S}(x)=\begin{piecewise}
        1 & x\in S\\
        0 & x\not\in S
    \end{piecewise}
$$

\begin{theorem}
    The characteristic function for the integers is given by:
    $$
        \mathbf{1}_{\mathbb{Z}}(x)=1-(\ceil{x}-\floor{x})
    $$

    \begin{proof}
        We use our piecewise notation for the ceiling and floor functions for convenience (where for the ceiling function we've just reindexed $n\to n+1$ in order to match the conditions):
        $$
            \ceil{x}-\floor{x}=\pwobj{n+1}{x\in(n,n+1]}{n\in\mathbb{Z}}-\pwobj{n}{x\in[n,n+1)}{n\in\mathbb{Z}}
        $$

        We can separate out the $x=n$ and $x=n+1$ pieces in each:
        \begin{align*}
            \ceil{x}-\floor{x} = &\pwobj{n+1}{x\in(n,n+1)}{n\in\mathbb{Z}}\cup\\
                                    &\pwobj{n+1}{x=n+1}{n\in\mathbb{Z}}\\
                               &-\pwobj{n}{x\in(n,n+1)}{n\in\mathbb{Z}}\cup\pwobj{n}{x=n}{n\in\mathbb{Z}}
        \end{align*}

        Reindexing the $x=n+1$ pieces with $n+1\to n$, and then combining all pieces for $x\in(n,n+1)$, we have that:
        \begin{align*}
            \ceil{x}-\floor{x} &= &\pwobj{1}{x\in(n,n+1)}{n\in\mathbb{Z}}\cup\pwobj{n}{x=n}{n\in\mathbb{Z}}\\
                               &&-\pwobj{n}{x=n}{n\in\mathbb{Z}} \\
                               &= &\pwobj{1}{x\in(n,n+1)}{n\in\mathbb{Z}}\cup\pwobj{0}{x=n}{n\in\mathbb{Z}}\\
        \end{align*}

        That is, for all integers $n$, if $x=n$ (i.e. $x$ is an integer) we have $\ceil{x}-\floor{x}=0$. Otherwise, for all non-integer $x$, we have $\ceil{x}-\floor{x}=1$. We therefore have that:
        \begin{align*}
            \ceil{x}-\floor{x} &=\begin{piecewise}
                1 & x\not\in\mathbb{Z} \\
                0 & x\in\mathbb{Z}
            \end{piecewise} \\
            &=-\begin{piecewise}
                -1 & x\not\in\mathbb{Z} \\
                0 & x\in\mathbb{Z}
            \end{piecewise} \\
            &=1-\begin{piecewise}
                0 & x\not\in\mathbb{Z} \\
                1 & x\in\mathbb{Z}
            \end{piecewise} \\
            &=1-\mathbf{1}_{\mathbb{Z}}(x)
        \end{align*}
    \end{proof}
\end{theorem}

Iverson bracket notation is almost a generalisation to these characteristic or indicator functions, except instead of using a set and variable, they are provided with any sort of predicate, condition, etc.:
$$
    [S]=\begin{piecewise}
        1 & S \\
        0 & \lnot S
    \end{piecewise}
$$

Essentially all piecewise functions we work with in these notes (to a point) are able to be written in terms of Iverson brackets. In turn, certain Iverson brackets can be written in terms of elementary functions or functions we've already seen.
\begin{theorem}
    We can write several of the previous functions in this section in terms of Iverson brackets:
    \begin{enumerate}
        \item $\mathbf{1}_{S}(x)=[x\in S]$ (noting that the complement of $x\in S$ is $x\in\mathbb{R}\setminus S$, for example)
        \item $\step{x}=[x>0]$
        \item $\sgn{x}=[x>0]-[x<0]$
        \item $\abs{x}=x[x>0]-x[x<0]$
        \item $\max\br{a,b}=a[a\geq b]+b[a<b]$
        \item $\min\br{a,b}=b[a\geq b]+a[a<b]$
    \end{enumerate}

    Proofs of these equivalences are left specifically for the reader to practice.
\end{theorem}

\begin{theorem}
    \label{theorem:iverson_form}
    Even more generally, let us denote the piecewise object:
    $$
        \phi=\pwobj{\varphi_i}{C_i}{i\in I}
    $$

    Such that $C_i\land C_j\leftrightarrow\bot$ for all $i\neq j$ and $i,j\in I$. Then $\phi$ can be expressed using Iverson bracket notation:
    $$
        \phi=\sum_{i\in I}{\varphi_i[C_i]}
    $$

    \begin{proof}
        Let us begin by splitting up the piecewise object $\phi$ into separate piecewise objects such that all but 1 piece is equal to $0$, in each:
        $$
            \pwobj{\varphi_i}{C_i}{i\in I}=\sum_{i\in I}{\pwobj{\varphi_m}{C_m}{m=i}\cup\pwobj{0}{C_n}{n\in I\setminus\br{i}}}
        $$

        Since each $C_n$ is distinct, then $C_n\leftrightarrow\lnot C_i$ for $n\neq i$. Therefore:
        $$
            \pwobj{\varphi_i}{C_i}{i\in I}=\sum_{i\in I}{\pwobj{\varphi_m}{C_m}{m=i}\cup\pwobj{0}{\lnot C_i}{n\in I\setminus\br{i}}}
        $$

        Since each piecewise object has two pieces only, we can rewrite this as:
        $$
            \pwobj{\varphi_i}{C_i}{i\in I}=\sum_{i\in I}{\begin{piecewise}
                \varphi_i & C_i \\
                0 & \lnot C_i
            \end{piecewise}}
        $$

        Then, factoring $\varphi_i$ and writing in terms of Iverson brackets, we have our result.
    \end{proof}
\end{theorem}
\newpage